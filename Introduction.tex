\chapter*{Introduction}
The Standard Model (SM) of Particle Physics is an effective theory supported by several experimental observations. The SM gathers quantum field theories which describe the elementary particles building the ordinary matter and how they interact via three mechanisms: the strong nuclear, the weak nuclear and the electromagnetic interactions. The strong and weak interactions are constrained to the atomic nuclei radius ($10^{-14}$ m) and are the responsible for keeping protons together and nuclei decaying processes, for instance. The electromagnetic interaction is the reason why opposite charged particles attract each other, for instance. Each of these interactions is mediated by particles called bosons. The strong nuclear interaction is mediated by gluons ($g$) which are particles without mass and that can interact with themselves. The weak nuclear interaction is mediated by massive bosons $Z$ and $W^{\pm}$. The electromagnetic interaction is mediated by the photon ($\gamma$) which does not have electric charge, neither mass \cite{bib:whitbeck-2013, bib:brachem-2012, bib:griffiths-2008, bib:halzen-martin-1984}.

One of the biggest mysteries of the SM is the mass of the elementary particles. According to the theory, in order to keep certain symmetry conditions (which are needed since Physics must be invariant under frame variation) the elementary particles have been initially treated as non-massive objects. Around 1964, though, an idea for spontaneously generate the particles mass in the theory without breaking its symmetry was conceived. Nowadays such idea is known as the \textit{Higgs mechanism}. This mechanism is associated to a new boson (the Higgs boson), which was the last missing piece on the SM basis for many years. The search for it was one of the motivations for the building of the LHC (\textit{Large Hadron Collider}). Finally, in 2012 a SM Higgs boson-like was observed by ATLAS and CMS collaborations, opening a new chapter to elucidate the origin of particles' mass \cite{bib:whitbeck-2013, bib:brachem-2012, bib:griffiths-2008, bib:halzen-martin-1984}.

After the Higgs boson discovery several measurements of its properties have been done and so far they show good compatibility with SM expectations. However, there is still room for beyond SM effects in the Higgs boson sector. An interesting channel is its production through Vector Boson Fusion (VBF). The VBF is the second most important production mode of Higgs bosons and presents a physic process clean of beyond SM effects at low-order (most probable diagrams). Hence it is an interesting channel to check the expectation from the SM. The Higgs produced via VBF followed by the presence of two energetic jets separated by a $\eta$ gap (a detector region), which is often used as a tagging input. In this context, this analysis presents the results of combining the power of Artificial Neural Networks (ANNs) and the extra information in the events given by a 3$^{rd}$ jet passing some required selections. An isolated measurement on the Higgs VBF signal strength is then made on dedicated subsets of events, selected in a defined signal region and orthogonally divided into two jet-based categories.