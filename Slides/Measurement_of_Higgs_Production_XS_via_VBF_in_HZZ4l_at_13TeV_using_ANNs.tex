\documentclass[9pt,serif,english]{beamer}
\setbeamersize{text margin left=3mm,text margin right=3mm}
\usepackage{ragged2e}
\usepackage[utf8x]{inputenc}
\usepackage{graphicx,color,colortbl}%Inclusao de graficos
\usepackage{subfig}
\usepackage[percent]{overpic}
\usepackage{multicol}
%\setlength{\columnsep}{-0.3cm}
\usepackage{hyperref}
\usepackage{cite}
\usepackage{multimedia}

\usetheme{CambridgeUS}%Theme ()
\usepackage[T1]{fontenc}
\usepackage[sc]{mathpazo}

\setbeamertemplate{footline}[frame number]

%--- define cor dos quadros no slide ------
\xdefinecolor{dark_red}{rgb}{0.7,0,0}
\xdefinecolor{light_gray}{rgb}{0.95,0.95,0.95}
\xdefinecolor{light_red}{rgb}{0.96,0.8,0.8}
\xdefinecolor{light_blue}{rgb}{0.8,0.8,0.96}
\usecolortheme[named=dark_red]{structure}
\setbeamercolor{title}{fg=white}
\setbeamercolor{title}{bg=dark_red}
\setbeamercolor{block title}{fg=white}
\setbeamercolor{block title}{bg=dark_red}
\setbeamercolor{block body}{bg=light_gray}

\newenvironment<>{varblock}[2][.9\textwidth]{%
\setlength{\textwidth}{#1}
\begin{actionenv}#3%
	\def\insertblocktitle{#2}%
	\par%
	\usebeamertemplate{block begin}}
	{\par%
	\usebeamertemplate{block end}%
\end{actionenv}}
\raggedbottom
%\usepackage{enumitem}

%my own template to include slide title
\setbeamertemplate{frametitle}{%
	\nointerlineskip%
	\begin{beamercolorbox}[wd=\paperwidth,ht=1.8ex,dp=0.3ex]{block body}
		\hspace*{1ex}\insertframetitle%
	\end{beamercolorbox}%
}

\begin{document}

\title{Measurement of Higgs Production Cross Section\\ via Vector Boson Fusion in $H \rightarrow ZZ \rightarrow 4l$ final state\\ at 13 TeV using Artificial Neural Networks}

\author{Miquéias Melo de Almeida$^{1}$\\[0.5cm]
		Advisor: Andre Sznajder$^{1}$\\
		Co-advisor: Nicola De Filippis$^{2}$}

\institute{\small $^{1}$Universidade do Estado do Rio de Janeiro\\$^{2}$INFN and Politecnico di Bari}
\date{April 11$^{th}$, 2019}

\titlegraphic{
	\centering
	\includegraphics[scale=0.15]{uerj_logo}
	\hspace{3cm}
	\includegraphics[scale=0.022]{infn_logo}
	\hspace{3cm}
	\includegraphics[scale=0.108]{cms_logo}
}

\frame{\titlepage}

\begin{frame}{\hspace{4.5cm} Table of Contents}
\tableofcontents
\end{frame}

%-------------------------------------------------------------
\section{CMS AN-18-120}
\begin{frame}
	\center
	\begin{minipage}{10cm}
		\begin{varblock}[5cm]{}
			\center
			\huge \color{red} The CMS AN-18-120
		\end{varblock}
	\end{minipage}
\end{frame}

\subsection{Introduction}
\begin{frame}{Introduction}
\begin{multicols}{2}
\begin{itemize}
		\justifying
	\footnotesize
	\item This analysis is a measurement of the Higgs VBF production XS in the HZZ4L channel using ANN discriminants;
	\item Analysis characteristics:
	\begin{itemize}
		\justifying
		\item follows similar requirements established in CMS HZZ4L analysis;
		\item VBF signal region (VBF-SR) defined similarly to CMS VBF category (no MELA);
		\item proposes the usage of a 3$^{rd}$ jet when available;
		\item events at VBF-SR divided into two jet-based subcategories;
		\item Artificial Neural Network (ANN) as a VBF discriminant;
	\end{itemize}
\end{itemize}
\begin{overpic}
	[scale=0.4]{figs/AN-18-120-cover}
	\put(16,103){\color{red}Documentation}
\end{overpic}
\end{multicols}
\end{frame}

\subsection{Theoretical Motivation}
\begin{frame}{Particle Physics and the Standard Model}
	\footnotesize
	\setlength\columnsep{-2.5cm}
	\begin{multicols}{2}
		\begin{overpic}
			[scale=0.37]{figs/matter_struct}
			\put(21,15){($>10^{-3}m$)}
			\put(17,35){($\sim10^{-8}m$)}
			\put(25,52){($\sim10^{-10}m$)}
			\put(5,85){($\sim10^{-14}m$)}
			\put(30,100){($<10^{-19}m$)}
		\end{overpic}
		\begin{minipage}{7.5cm}
			\begin{itemize}
		\justifying
				\item Elementary Particle Physics is the field on Physics dedicated to the study of fundamental building blocks of matter and their interactions;
				\item The Standard Model (SM) resumes what physicists know so far:
			\end{itemize}
			\includegraphics[scale=0.25]{figs/sm}
		\end{minipage}	
	\end{multicols}
\end{frame}

\begin{frame}{The Higgs Boson}
	\footnotesize
	\begin{multicols}{2}
		\begin{itemize}
		\justifying
			\item In Physics symmetry dictates interactions (gauge theories);
			\item However, gauge bosons and fermions must be massless in the EWK theory;
			\item Solution: Higgs scalar field to spontaneously break gauge symmetry giving mass to W, Z and fermions;
			\item Higgs particle (the field quanta) can be produced in some process, such as:
			\begin{overpic}
				[scale=0.25]{figs/higgs_plot}
				\put(25,50){\color{red}\small Higgs}
				\put(15,84){Higgs discovered at the LHC in 2012}
			\end{overpic}
		\end{itemize}
	\end{multicols}
	\vspace{0.2cm}
	\begin{multicols}{4}
		\begin{overpic}
			[scale=0.04]{figs/ggh_diagram}
			\put(20,87){$gg \rightarrow H$}
			\put(10,-17){fusion of gluons}
		\end{overpic}
		\begin{overpic}
			[scale=0.04]{figs/vbf_diagram}
			\put(15,90){$q\bar{q} \rightarrow Hq\bar{q}$}
			\put(-10,-17){fusion of bosons Z/W}
			\put(30,-30){(VBF)}
		\end{overpic}
		\begin{overpic}		
			[scale=0.04]{figs/vh_diagram}
			\put(15,105){$q\bar{q} \rightarrow VH$}
			\put(-10,-17){radiation from Z/W}
		\end{overpic}
		\begin{overpic}		
			[scale=0.04]{figs/tth_diagram}
			\put(20,90){$gg \rightarrow t\bar{t}H$}
			\put(-10,-15){fusion of top quarks}
		\end{overpic}								
	\end{multicols}
\end{frame}

\begin{frame}{Physics Processes in this Analysis}
	\begin{multicols}{2}
		\begin{minipage}{9cm}
			\begin{itemize}
		\justifying
				\item Signal is Vector Boson Fusion (VBF):
				\begin{itemize}
		\justifying
					\item second largest Higgs production mode;
					\item tree-level and clean of beyond-SM processes;
					\item most theoretically precise production mode;
					\item good frame for probing SM electroweak sector;
					\item estimated from MC normalized by $\sigma$.$BR$ given by LHC Higgs XS Working Group;
				\end{itemize}
			\end{itemize}
		\end{minipage}
		\flushright
		\begin{minipage}{2.8cm}
			\vspace{-0.5cm}
			\begin{varblock}[2.8cm]{\center VBF (only)}
				\begin{figure}
					\subfloat[$q\bar{q} \rightarrow Hq\bar{q}$]{\includegraphics[scale=0.04,trim={0cm 0cm 0cm 1cm},clip]{figs/vbf_diagram}}
				\end{figure}
			\end{varblock}		
		\end{minipage}		
	\end{multicols}
	\vspace{-0.7cm}
	\begin{itemize}
		\justifying
		\item Backgrounds:
	\end{itemize}
	\begin{multicols}{2}
		\begin{varblock}[6.5cm]{\center Remaining SM Higgs Production Modes}
			\center
			Estimated from MC normalized by $\sigma$.$BR$ given by LHC Higgs XS Working Group\\[-0.4cm]
			\begin{figure}
				\subfloat[$gg \rightarrow H$]{\includegraphics[scale=0.04]{figs/ggh_diagram}}
				\subfloat[$q\bar{q} \rightarrow VH$]{\includegraphics[scale=0.04]{figs/vh_diagram}}
				\subfloat[$gg \rightarrow t\bar{t}H$]{\includegraphics[scale=0.04]{figs/tth_diagram}}
			\end{figure}
		\end{varblock}
		\flushright
		\begin{minipage}{4.5cm}
			\vspace{-0.2cm}   		
			\begin{varblock}[4.5cm]{\center ZZ's}
				\center
				Estimated from MC normalized by $\sigma$.$BR$\\[-0.7cm]
				\begin{figure}
					\subfloat[$q\bar{q} \rightarrow ZZ$]{\includegraphics[scale=0.04,trim={0cm 3cm 0cm 4cm},clip]{figs/qqzz_diagram}}
					\subfloat[$gg \rightarrow ZZ$]{\includegraphics[scale=0.04,trim={0cm 3cm 0cm 4cm},clip]{figs/ggzz_diagram}}
				\end{figure}
			\end{varblock}
		\end{minipage}   	
	\end{multicols}
	\vspace{-0.4cm}
	\begin{block}{\center $Z+X$}
		\center
		Estimated via Fake Rate \textit{data-driven} method
	\end{block}   	
\end{frame}

\subsection{The CMS Experiment at CERN}
\begin{frame}{{\color{blue}C}ompact {\color{blue}M}uon {\color{blue}S}olenoid in a Nutshell}
	\vspace{-0.5cm}
	\begin{multicols}{2}
		\includegraphics[width=8.6cm,height=5.2cm]{figs/cms_scheme}\\
		\flushright
		\includegraphics[width=3.3cm,height=2.6cm]{figs/muon_csc_dt}\\
		\includegraphics[width=3.3cm,height=2.6cm]{figs/muon_barrel}
	\end{multicols}
	\vspace{-0.3cm}
	\includegraphics[width=3.83cm,height=2.7cm]{figs/strip_tracker}\quad
	\includegraphics[width=3.83cm,height=2.7cm]{figs/ecal_mont}\quad
	\includegraphics[width=3.83cm,height=2.7cm]{figs/cms_hcal}
\end{frame}

\begin{frame}{CMS-Particles Interaction Profile}
	\center
	\begin{overpic}
		[scale=0.3]{figs/cms_deteccao}
		\put(30,51){Particles Profiles through CMS Detector}
	\end{overpic}
	\begin{itemize}
		\justifying
		\item CMS has particle-specialized sub-detectors;
		\item "Long-life" particles (reach detectors) are identified by signal patterns on CMS;
		\item "Short-life" particles (decay into other particles, not reaching any detector) are identified through the properties of "long-life" particles;
	\end{itemize}	
\end{frame}

\subsection{Datasets, Triggers and Simulated Samples}
\begin{frame}{Datasets and Triggers}
	\begin{itemize}
		\justifying
		\item Data:
		\begin{itemize}
		\justifying
			\item full 2016 Data: L = 35.9fb$^{-1}$, 03Feb ReReco (full list in backup);
			%\item JSON: Cert\_271036\_284044\_13TeV\_23Sep2016ReReco\_Collisions16\_JSON.txt.
		\end{itemize}
		\item Triggers:
		\begin{itemize}
		\justifying
			\item based on multi-lepton HLT paths;
			\item isolated di-lepton paths + non-isolated tri-lepton paths + single-lepton paths;
			\item requirements to avoid double-counting is applied;
			\item overall trigger efficiency is higher than 99$\%$ wrt. 4-lepton analysis selection;
		\end{itemize}
	\end{itemize}
	\centering
	\includegraphics[scale=0.4]{figs/trigger_paths}
\end{frame}

\begin{frame}{Simulated Samples}
	\begin{itemize}
		\justifying
		\footnotesize
		\item Signal: VBF\_HToZZTo4L\_M125\_13TeV\_powheg2\_JHUgenV6\_pythia8;
		\item Background: remaining samples in the table;
		\item Special samples for ggH ({\color{blue}*}) and qqZZ ({\color{blue}**}) included.
	\end{itemize}
	\centering
	\begin{overpic}
		[scale=0.5]{figs/simulated_samples}
		\put(0,61){\color{blue}*}
		\put(-1,32){\color{blue}**}
	\end{overpic}
\end{frame}

\subsection{Objects and Events Selections}
\begin{frame}{Objects Selections}
	\begin{block}{\center Electrons (momentum callibration applied)}
		\hspace{0.35cm}
		\begin{minipage}{9cm}
		\begin{multicols}{2}
			\begin{varblock}[4cm]{\center Loose}
				\begin{itemize}
		\justifying
					\item $p_{T} > 7$ GeV;
					\item $|\eta| < 2.5$;
					\item $|d_{xy}| < 0.5$ cm;
					\item $|d_{z}| < 1.0$ cm.
				\end{itemize}				
			\end{varblock}
			\begin{varblock}[6.5cm]{\center Tight}
				\begin{itemize}
		\justifying
					\item Loose selections plus:
					\begin{itemize}
		\justifying
						\item $|SIP_{3D}| < 4.0$;
						\item Isolation ($\Delta R = 0.3$) < 0.35;
						\item MVA (BDT) calorimeter-based:
					\end{itemize}
				\end{itemize}
				\centering
				\includegraphics[scale=0.35]{figs/electron_bdt}
			\end{varblock}
		\end{multicols}
	\end{minipage}
	\end{block}
	\begin{block}{\center Muons (momentum callibration and FSR applied)}
		\hspace{0.35cm}
		\begin{minipage}{9cm}
			\begin{multicols}{2}
				\begin{varblock}[4cm]{\center Loose}
					\begin{itemize}
		\justifying
						\item Global/Tracker Muons;
						\item $p_{T} > 5$ GeV;
						\item $|\eta| < 2.4$;
						\item $|d_{xy}| < 0.5$ cm;
						\item $|d_{z}| < 1.0$ cm.
					\end{itemize}
				\end{varblock}
				\begin{varblock}[6.5cm]{\center Tight}
					\begin{itemize}
		\justifying
						\item Loose selections plus:
						\begin{itemize}
		\justifying
							\item PF Muon;
							\item $|SIP_{3D}| < 4.0$;
							\item Isolation ($\Delta R = 0.3$) < 0.35;
							\item Ghost-cleaning (single-$\mu$ as more).
						\end{itemize}
					\end{itemize}				
				\end{varblock}
			\end{multicols}
		\end{minipage}
	\end{block}	
\end{frame}

\begin{frame}{Objects Selections}
	\begin{block}{\center Photons and FSR}
		\begin{itemize}
		\justifying
			\item $p_{T} > 2$ GeV, $|\eta| < 2.4$;
			\item Isolation ($\Delta R = 0.3$) < 1.8;
			\item $\Delta R (\gamma,l)/E_{T,\gamma}^{2} \geq 0.012$ and $\Delta R (\gamma,l) \geq 0.5$;
		\end{itemize}
	\end{block}
	\begin{block}{\center Jets (JECs applied)}
		\begin{itemize}
		\justifying
			\item anti-kT (R = 0.4) PF CHS (\textit{Carged Hadron Subtracted}) jets;
			\item $p_{T} > 30$ GeV, $|\eta| < 4.7$;
			\item cleaning $\Delta R(jet,l/\gamma) > 0.4$;
			\item b-tagging with CSV (\textit{Combined Secondary Vertex}) algorithm;
		\end{itemize}
	\end{block}	
	\begin{block}{\center MET (\textit{Missing Transverse Energy})}
		\begin{itemize}
		\justifying
			\item PF MET with type-1 correction:
			$\vec{E}_{T}^{miss} = - ( \sum_{jets} \vec{p}_{T}^{JEC} + \sum_{uncl.} \vec{p}_{T} )$;
			\item filters\footnote{See backup slides.} from JETMET POG applied (improves signal-to-noise ratio).
		\end{itemize}
	\end{block}	
\end{frame}

\begin{frame}{Event Selections}
	\begin{block}{\center SM Higgs Region}
		\begin{itemize}
		\justifying
			\item[1] \textbf{Z candidates}: pair of same-flavor, opposite-charge and FSR corrected leptons ($e^{+}e^{-}$, $\mu^{+}\mu^{-}$) having invariant mass $12 < m_{ll(\gamma)} < 120$ GeV;
			\item[2] \textbf{ZZ candidates}: pair of non-overlapping (different leptons) Z candidates. The Z with smallest $|m_{ll(\gamma)}-m_{Z}^{PDG}|$ is identified as $Z_{1}$ and the other Z as $Z_{2}$. ZZ candidates must satisfy:
			\begin{itemize}
		\justifying
				\item any two leptons must have $\Delta R(\eta,\phi) > 0.02$ (\textbf{ghost removal});
				\item at least two out of the four leptons must have $p_{T} >$ 10 and 20 GeV;
				\item any two leptons must have (without FSR-$\gamma$) $m_{ll} >$ 4 GeV (\textbf{QCD suppression});
				\item $m_{Z_{1}} >$ 40 GeV and $m_{Z_{1}Z_{2}} >$ 100 GeV;
				\item if more than one ZZ candidate survives previous cuts, the one with highest scalar leptons $p_{T}$ sum is chosen;
			\end{itemize}
		\end{itemize}
	\end{block}
\end{frame}

\begin{frame}{m4l Distributions and Yields}
		\begin{block}{\centering After SM Higgs Selections}
			\centering
			\includegraphics[scale=0.26,trim={3cm 1cm 2cm 1cm},clip]{figs/m4l_smhiggs_4l_full_range}
			\includegraphics[scale=0.28]{figs/smhiggs_yields_table}
		\end{block}	
\end{frame}

\begin{frame}{Event Selections}
	\begin{block}{\center SM Higgs Region}
		\begin{itemize}
			\justifying
			\item[1] \textbf{Z candidates}: pair of same-flavor, opposite-charge and FSR corrected leptons ($e^{+}e^{-}$, $\mu^{+}\mu^{-}$) having invariant mass $12 < m_{ll(\gamma)} < 120$ GeV;
			\item[2] \textbf{ZZ candidates}: pair of non-overlapping (different leptons) Z candidates. The Z with smallest $|m_{ll(\gamma)}-m_{Z}^{PDG}|$ is identified as $Z_{1}$ and the other Z as $Z_{2}$. ZZ candidates must satisfy:
			\begin{itemize}
				\justifying
				\item any two leptons must have $\Delta R(\eta,\phi) > 0.02$ (\textbf{ghost removal});
				\item at least two out of the four leptons must have $p_{T} >$ 10 and 20 GeV;
				\item any two leptons must have (without FSR-$\gamma$) $m_{ll} >$ 4 GeV (\textbf{QCD suppression});
				\item $m_{Z_{1}} >$ 40 GeV and $m_{Z_{1}Z_{2}} >$ 100 GeV;
				\item if more than one ZZ candidate survives previous cuts, the one with highest scalar leptons $p_{T}$ sum is chosen;
			\end{itemize}
		\end{itemize}
	\end{block}	
	\centering {\huge \textbf{+}}
	\begin{block}{\center VBF Signal Region (VBF-SR)}
		\begin{itemize}
			\justifying
			\item[3] In order to enhance VBF-to-background ratio:
			\begin{itemize}
				\justifying
				\item Number of jets:
				\begin{itemize}
					\justifying
					\item EITHER, 2 or 3 jets from which at most one b-tagged jet;
					\item OR, more than 3 jets with no b-tagged jet;
				\end{itemize}
				\item ZZ candidates must have $118 \leq m_{Z_{1}Z_{2}} \leq 130$ GeV;
			\end{itemize}
		\end{itemize}
	\end{block}
\end{frame}

\begin{frame}{$m_{4l}$ Distributions and Yields}
		\begin{block}{\centering After VBF-SR Selections}
			\centering
				\includegraphics[scale=0.26,trim={3cm 1cm 2cm 1cm},clip]{figs/m4l_vbf_4l_118_130GeV}
				\includegraphics[scale=0.28]{figs/vbf_yields_table}
			\begin{itemize}
				\item The MC and Z+X remaining at VBF-SR are the events used in order to train and test the ANN models;
			\end{itemize}
		\end{block}
\end{frame}

\begin{frame}{$Z+X$ Background (Data-Driven) Estimation}
	\begin{itemize}
		\justifying
		\item $Z+X$ originates from processes with non-prompt leptons: heavy-flavor meson decays, mis-reconstructed jets and electrons from $\gamma$ conversions;
		\item Strategy: measure FR in specific control regions (CRs) and apply it to the SR;
		\item {\color{red}First step}, measuring the FR:
		\begin{itemize}
		\justifying
			\item samples of $Z_{l_{1}l_{2}}+l_{3}$ ($l_{1,2}$ \textit{tight} leptons, $l_{3}$ \textit{loose} lepton);
			\item $p_{T}^{l_{1},l_{2}} >$ 10, 20 GeV, $m_{l_{1}l_{2}} <$ 4 GeV, $|m_{l_{1}l_{2}}-m_{Z}^{PDG}| <$ 7 GeV and $E_{T}^{miss} <$ 25 GeV;
			\item contribution from $WZ$ (with potential 3 real leptons) is subtracted;
		\end{itemize}
		\item The FR ($N_{tight}/N_{loose}$, ie. probability of \textit{loose} lepton pass \textit{tight} selections) is mapped wrt. to $p_{T}^{l_{3}}$ vs. $\eta^{l_{3}}$:
	\end{itemize}
	\centering
	\includegraphics[scale=0.28]{figs/fake_rate_2D_maps_corrected}
\end{frame}

\begin{frame}{$Z+X$ Background (Data-Driven) Estimation}
	\begin{itemize}
		\justifying
		\item {\color{red}Second step}, building CRs:
		\begin{itemize}
		\justifying
			\item Orthogonal to the SM Higgs selections and enriched by fake-lepton events;
			\item Require $Z_{l_{1}l_{2}}Z_{l_{3}l_{4}}$ where $l_{1,2}$ are always \textit{tight} leptons and if $l_{3,4}$ are \textit{loose} leptons, define 2P2F while if only $l_{4}$ is \textit{loose}, define 3P1F;
		\end{itemize}
	\end{itemize}
	\vspace{-0.3cm}
	\begin{multicols}{2}
	\begin{varblock}[5.8cm]{\center 2P2F CR: $w_{Data} = \frac{f_{3}}{1-f_{3}}+\frac{f_{4}}{1-f_{4}}$}
		\begin{overpic}
			[scale=0.135,trim={1cm 0cm 2cm 1cm},clip]{figs/m4l_2p2f_4mu_smhiggs}
		\end{overpic}\hspace{-0.1cm}
		\includegraphics[scale=0.135,trim={1cm 0cm 2cm 1cm},clip]{figs/m4l_2p2f_4e_smhiggs}\\
		\includegraphics[scale=0.135,trim={1cm 0cm 2cm 1cm},clip]{figs/m4l_2p2f_2e2mu_smhiggs}
		\includegraphics[scale=0.135,trim={1cm 0cm 2cm 1cm},clip]{figs/m4l_2p2f_2mu2e_smhiggs}		
	\end{varblock}	
	\begin{varblock}[5.8cm]{\center 3P1F CR: $w_{Data} = \frac{f_{4}}{1-f_{4}}$}
		\begin{overpic}
			[scale=0.135,trim={1cm 0cm 2cm 1cm},clip]{figs/m4l_3p1f_4mu_smhiggs}
		\end{overpic}\hspace{-0.1cm}
		\includegraphics[scale=0.135,trim={1cm 0cm 2cm 1cm},clip]{figs/m4l_3p1f_4e_smhiggs}\\
		\includegraphics[scale=0.135,trim={1cm 0cm 2cm 1cm},clip]{figs/m4l_3p1f_2e2mu_smhiggs}
		\includegraphics[scale=0.135,trim={1cm 0cm 2cm 1cm},clip]{figs/m4l_3p1f_2mu2e_smhiggs}		
	\end{varblock}		
	\end{multicols}	
\end{frame}

\begin{frame}{$Z+X$ Background (Data-Driven) Estimation}
	\begin{itemize}
		\justifying
		\item {\color{red}Third step}, using the measured FR($p_{T}$, $\eta$) in order to estimate $Z+X$ yield and shape in the SR;
		\item At SR, $Z+X$ is given by two components: one from 2P2F and one from 3P1F, via the observed Data and ZZ contribution;
		\item Events selected in the two CRs were stored for training ANN and derive its $Z+X$ shape;
	\end{itemize}
	\centering
	\begin{minipage}{8.2cm}
		\begin{varblock}[8.2cm]{\centering $Z+X$ at SR: $N^{bkg}_{SR} = (1 - \frac{N^{ZZ}_{3P1F}}{N_{3P1F}}) \sum^{N_{3P1F}}_{i} \frac{f^{i}_{a}}{(1-f^{i}_{a})} - \sum^{N_{2P2F}}_{j} \frac{f^{j}_{b}}{(1-f^{j}_{b})} \frac{f^{j}_{c}}{(1-f^{j}_{c})}$}
				\begin{overpic}
					[scale=0.18,trim={0.5cm 0cm 1cm 0cm},clip]{figs/m4l_zx_4l_smhiggs}
					\put(37,85){SM Higgs}
				\end{overpic}
				\begin{overpic}
					[scale=0.18,trim={0.5cm 0cm 2cm 0cm},clip]{figs/m4l_zx_4l_vbf}
					\put(37,85){VBF-SR}
				\end{overpic}
		\end{varblock}
	\end{minipage}		
\end{frame}

\subsection{Artificial Neural Networks (ANNs)}
\begin{frame}{Artificial Neural Networks (ANNs)}
	\begin{itemize}
		\item An ANN is a composite of multidimensional parameterized functions in terms of weights ($w_{i}$) and biases ($b_{i}$): $\Omega(x) =  \Phi(f_{1}(w_{1}, b_{i}; x),..., f_{n}(w_{n}, b_{n}; x))$;
	\end{itemize}
	\center		
	\begin{overpic}[scale=0.22]{figs/neuron}
		\put(60,49){\footnotesize \color{blue}ANN neuron}
	\end{overpic}
	\quad
	\quad
	\begin{overpic}[scale=0.24]{figs/DNN}
		\put(0,65){\footnotesize\color{blue}Assembly many neurons to build a net}
		\put(25,0){\footnotesize\color{red}Depth reduces width}
	\end{overpic}	
	\begin{itemize}
		\justifying
		\item It comes from Machine Learning field (closest one to AI's development) in Computer Science;
		\item ANNs have been successfully applied in many tasks as recognition of hand-written digits, images, sounds, sequences of data, etc;
		\item Google and Microsoft are investing on them: Deep Mind, Cloud Machine Learning Engine, Cortana, Azure Studio, etc;
	\end{itemize}
\end{frame}	

\begin{frame}{ANN Supervised Learning}
	\vspace{-0.1cm}	
	\begin{itemize}
		\justifying
		\item Training an ANN consists in finding parameters ($w_{i}$ and $b_{i}$) for $\Omega(x)$, such that it models a dataset in the best (possible) way;
		\item Supervised ANN training:
		\begin{itemize}
			\item[1] Initialize ANN weights ($w_{ji}, b_{ji}$) randomly;
			\item[2] Input at the ANN an arbitrary set (\textbf{x}, $\Omega($\textbf{x}$)$);
			\item[3] Compute the ANN error: (Loss) $\mathcal{E} = f(\Omega(\textbf{x})-\Omega(\textbf{x})_{pred})$;
			\item[4] Update the neuron parameters with loss gradient descent: $\Delta w_{ji} = - \eta \dfrac{\partial \mathcal{E}}{\partial w_{ji}}$;
			\item[5] Repeat from 2 to 3 until $\mathcal{E}$ gets smaller enough;
		\end{itemize}
		%\vspace{-0.3cm}
		%\begin{multicols}{2}
		%\begin{itemize}
		%	\item ANN receives \textbf{x}'s and expected $\Omega(x)$;
		%	\item Computes difference between expected and ANN prediction;
		%	\item Update $w_{i}$ and $b_{i}$ using this difference;
		%	\item Repeat until predictions are good enough (minimize the difference);
		%\end{itemize}
		%\flushleft
		%\begin{overpic}
		%	[width=4cm,height=2cm]{figs/ann_supervised_learning}
		%	\put(-5,37){\textbf{x}}
		%	\put(102,35){\scriptsize $\Omega($\textbf{X})$_{pred}$}
		%	\put(92,45){\scriptsize $\mathcal{E}(\Omega_{exp}-\Omega_{pred})$}
		%	\put(28,25){\tiny $\Delta w_{ji} (n) \propto \eta \frac{\partial \mathcal{E}}{\partial w_{ji}}$}
		%	\put(30,42){\scriptsize $w_{ji}(n)$, $b_{ji}(n)$}
		%	\put(22,4){\scriptsize $w_{ji}(n+1)$, $b_{ji}(n+1)$}
		%\end{overpic}
		%\end{multicols}
		%\vspace{-0.3cm}
		\item Here's an ANN learning $f(x) = a.x + b$ (just a linear dataset):
	\end{itemize}
	\centering
	\movie[height=4.6cm,width=11.8cm,loop,showcontrols]{
		\begin{overpic}
			[height=4.6cm,width=11.7cm]{figs/tonystark_jarvis}
			\put(42,35){\color{red}\huge{Wake up Jarvis...}}
			\put(42,31){\color{red}\huge{Let's entertain the audience!}}
		\end{overpic}
	}{figs/nn_learning_linear_data.mp4}\\
	\tiny \url{https://www.youtube.com/watch?v=XAF1skB_MUw}
\end{frame}	

\begin{frame}{ANN Learning Situations}
	\begin{itemize}
		\item An ANN training commonly leads to one of these three main situations:
	\end{itemize}
	\begin{multicols}{3}
		\begin{varblock}[3.7cm]{\center \textbf{Underfit}}
			\includegraphics[scale=0.4]{figs/underfit}
			\begin{itemize}
				\small
				\item too few parameters;
				\item not capable to learn data;
				\item Solution: more parameters (layers/neurons) or training time;
			\end{itemize}			
		\end{varblock}
		\begin{varblock}[3.7cm]{\center \textbf{Goodfit}}
			\includegraphics[scale=0.4]{figs/goodfit}
			\begin{itemize}
				\small
				\item enough parameters;
				\item learns data properly;
				\item capable of correct predictions for unseen data (generalization);
			\end{itemize}						
		\end{varblock}
		\begin{varblock}[3.7cm]{\center \textbf{Overfit}}
			\includegraphics[scale=0.4]{figs/overfit}
			\begin{itemize}
				\small
				\item too many parameters;
				\item learns data variance (noise);
				\item Solution: reduce parameters or training time, dropout, regularizations;
			\end{itemize}			
		\end{varblock}				
	\end{multicols}
\end{frame}

\begin{frame}{Monitoring the ANN Learning Process}
	\begin{itemize}
		\item The previous three cases can be monitored by looking at the so called learning curves;
		\item One defines at least two sub-datasets:
		\begin{itemize}
			\item training: which is used for training the ANN (seen data);
			\item testing: ANN make predictions on it at every epoch (unseen data);
		\end{itemize}
		\item This produces the following curves:
		\begin{multicols}{2}
			\includegraphics[scale=0.55,trim={2cm 0cm 1cm 1cm},clip]{figs/early_stop}\\
			\item As the model adapts to the seen data, it becomes worse to make correct predictions on new (unseen) data;
			\item When the validation loss becomes flat or increases while training loss keeps decreasing is a recommended procedure to stop the ANN training in order to avoid the overfitting;
		\end{multicols}
	\end{itemize}
\end{frame}

\begin{frame}{MC Preparation for ANN Training}
	\begin{itemize}
		\justifying
		\item For the ANN studies the MCs\footnote{Data is not used on ANN training.} are prepared in the following way:
		\item The channels (4e, 4$\mu$ and 2e2$\mu$ selected separately) are merged into just a sample for each MC (the channels are randomized inside the sample);
		\item Each merged sample are split into 2 independent sets:
		\begin{itemize}
		\justifying 
			\item \textbf{Training}: contains 80$\%$ of all events (from each sample) and is used to train ANNs;
			\item \textbf{Testing}: contains remaining 20$\%$ of events and is used to test ANN after training;
		\end{itemize}
		\item Then each set of all MCs are merged (and randomized) to compose the final training/testing input set to train/test ANNs;
		\item Additionally, two subsets have been defined based on the available number of jets per event:
		\begin{itemize}
		\justifying
			\item \textbf{Njets2}: only events with exactly two jets;
			\item \textbf{Njets3}: only events with at least three jets;
		\end{itemize}
		\item ANNs are built and trained via the open-source {\color{red}Keras}\footnote{{\color{blue}\url{https://keras.io/}} (now interfaced with TMVA).} (standard ML community tool) python package;
	\end{itemize}		
\end{frame}

\begin{frame}{Training Strategy}
	\begin{itemize}
		\justifying
		\item It's hard to assure a set of parameters as the best one;
		\item ANN architecture optimization by scanning over several parameters;
	\end{itemize}
	\begin{block}{\centering Training parameters (focus on low level variables)}
	\centering
	\begin{tabular}{c|l}
		\hline
		\textbf{Parameter}      & \textbf{Tested options}\\
		\hline
		Inputs         & leptons/jets($p_{T}$,$\eta$,$\phi$), MET\\
		\hline
		Pre-processing & none, normalization, standardization\\
		\hline
		Topologies\footnote{\tiny That refers to hidden layers. The output layer is always single sigmoid neuron}     & 7:5:3, 21:13:8, 10:10:10:10, 30, 100, ...\\
		\hline
		Early stop\footnote{\tiny Number of epochs to stop training if no improvement occurs.}     & 100, 600, 3000\\
		\hline
		Minimizer\footnote{\tiny Method used to compute parameters update.}      & SGD, Adam, Adagrad, Adadelta, RMSprop\\
		\hline
		Batch size\footnote{\tiny Subset from training set used to get parameters updates. N batches = N iterations per epoch.}     & 1, 5, 32, 64, 128, 786\\
		\hline
		Neuron         & ReLU, SeLU\\
		\hline
		Loss scaling\footnote{\tiny It's possible to use weights in training to optimize discrimination.}   & XS (process total XS), $\sigma.\epsilon.BR$ (event weight)\\
		\hline
		Dropout\footnote{\tiny Fraction of inputs randomly set to zero durging training.} & none, 0.1, 0.3, 0.5, 0.7, 0.9, 0.99, 0.3:0.4:0.2, 0.5:0.25:0.1\\
		\hline
	\end{tabular}		
	\end{block}
\end{frame}

\begin{frame}{Training Strategy}
	\begin{itemize}
		\justifying
		\item Trained ANN models has been validated by checking plots:
	\end{itemize}
	\centering
	\begin{tabular}{p{3.5cm}p{3.5cm}p{3.5cm}}
	\includegraphics[scale=0.22]{../ChapterAnalysis/figs/MELAOvertrainingCheck_example}&
	\includegraphics[scale=0.22]{../ChapterAnalysis/figs/NNOvertrainingCheck_example}&
	\includegraphics[scale=0.2]{../ChapterAnalysis/figs/ComparisonMCsEffPurity_example}\\
	\includegraphics[scale=0.21]{../ChapterAnalysis/figs/ComparisonMCsSignificance_example}&
	\includegraphics[scale=0.15]{../ChapterAnalysis/figs/FinalTrainTestROCs_example}&
	\includegraphics[scale=0.15]{../ChapterAnalysis/figs/ComparisonMCsROC_example}\\
	\end{tabular}
\end{frame}

\begin{frame}{View of Performances for Several ANN Parameters}
	\begin{itemize}
		\item Here are some performances of ANN parameters obtained during the analysis;
		\item This allowed to fix some parameters and make more focused trainings;
		\item The final ANNs have been chosen based on the metric present in this plots ($\epsilon.\pi$);
	\end{itemize}
	\begin{multicols}{2}
		\includegraphics[scale=0.26,trim={1.2cm 8.5cm 0cm 2.1cm},clip]{figs/SummaryOfResults_Metrics_Njets2_test_inputs}\\[0.3cm]
		\includegraphics[scale=0.26,trim={1.2cm 8.5cm 0cm 2.1cm},clip]{figs/SummaryOfResults_Metrics_Njets2_test_topology}\\[0.3cm]
		\includegraphics[scale=0.26,trim={1.2cm 8.5cm 0cm 2.1cm},clip]{figs/SummaryOfResults_Metrics_Njets2_test_minimizer}\\[0.3cm]
		\includegraphics[scale=0.26,trim={1.2cm 8.5cm 0cm 2.1cm},clip]{figs/SummaryOfResults_Metrics_Njets2_test_neuron}\\[0.3cm]
		\includegraphics[scale=0.26,trim={1.2cm 8.5cm 0cm 2.1cm},clip]{figs/SummaryOfResults_Metrics_Njets2_test_scaletrain}\\[0.3cm]
		\includegraphics[scale=0.26,trim={1.2cm 8.5cm 0cm 2.1cm},clip]{figs/SummaryOfResults_Metrics_Njets2_test_nooutliers}	
	\end{multicols}
\end{frame}

\begin{frame}{Final ANNs for Njets2, Njets3 and their Combination}
	\begin{itemize}
		\justifying
		\item Here are the final (best) ANNs, which have been chosen based on the $max(\epsilon.\pi)$ (ie. efficiency $\times$ purity);
	\end{itemize}
	\begin{multicols}{3}
		\begin{varblock}[3.8cm]{\centering Njets2}
			\includegraphics[scale=0.2,trim={3cm 1cm 3cm 1cm},clip]{figs/k57nj2_shapes_prefit}
		\end{varblock}
		\begin{varblock}[3.8cm]{\centering Njets3}
			\includegraphics[scale=0.2,trim={3cm 1cm 3cm 1cm},clip]{figs/k24nj3_shapes_prefit}
		\end{varblock}		
		\begin{varblock}[3.8cm]{\centering Njets2+Njets3}
			\includegraphics[scale=0.2,trim={3cm 1cm 3cm 1cm},clip]{figs/k57nj2_k24nj3_shapes_prefit}
		\end{varblock}				
	\end{multicols}
	\begin{itemize}
		\justifying
		\item $Z+X$ shape derived by feeding NNs with observed Data and $ZZ$ MC \footnote{Note that, the FR doesn't need to be redone.} and, repeating the procedure previously explained for this background estimation;
	\end{itemize}
\end{frame}

\begin{frame}{Why to use 3$^{rd}$ Jet?}
	\begin{multicols}{2}
		\begin{itemize}
			\justifying
			\item VBF topology is commonly tagged by two highly energetic jets with high $\eta$ separation;\\[0.1cm]
			\begin{minipage}{6cm}
				\item For the total VBF cross section, though, there are other Feynman diagrams;
				\item One of them is the case of a 3$^{rd}$ jet in the event which is irradiated from one of the two main jets (tree-level);
				\item Our studies show the 3$^{rd}$ jet in the events contains valuable information which allows to increase signal/background ratio;
			\end{minipage}
			\begin{table}
				Fraction of events, after the VBF-SR selections, containing the $jet_{i}$ (i-th jet)
				\centering
				\begin{tabular}{c|c|c}
					\hline
					\rowcolor{light_gray}
					& $j_{3}(\%)$ & $j_{4}(\%)$\\
					\hline
					$qqH$  & 21.3        & 5.0\\
					\hline
					$ggH$  & 28.8        & 7.1\\
					\hline
					$qqZZ$ & 19.5        & 2.9\\
					\hline
					$Data$ & 18.0        & 0.0\\
					\hline
				\end{tabular}
			\end{table}		
		\end{itemize}
		\begin{figure}
			VBF discrimination against the backgrounds using only 2 and up to 3 jets\\
			\centering
			\begin{overpic}
				[scale=0.27]{../ChapterAnalysis/figs/2jets_dnn_mela_djet_rocs_hjj}
				\put(12,85){\color{red}2jets}
				\put(45,26){\includegraphics[height=1.8cm,width=1.8cm]{../ChapterAnalysis/figs/2jets_sb_seff}}
			\end{overpic}
			\\
			\begin{overpic}
				[scale=0.27]{../ChapterAnalysis/figs/3jets_split0p8_norm_roc_djet_mela_dnn_hjj}
				\put(12,85){\color{red}3jets}
				\put(45,26){\includegraphics[height=1.8cm,width=1.8cm]{../ChapterAnalysis/figs/3jets_sb_seff}}
			\end{overpic}	
		\end{figure}
	\end{multicols}
\end{frame}

\subsection{Systematic Uncertainties}
\begin{frame}{Experimental and Theoretical Systematic Uncertainties}
	\begin{itemize}
		\justifying
		\item Experimental and Theoretical systematic uncertainties accounted in this analysis (enter as log-normal nuisance parameters in the statistical analysis):
	\end{itemize}	
	\begin{multicols}{2}
		\begin{varblock}[5.4cm]{\centering Experimental Uncertainties}
			\centering
			\includegraphics[scale=0.46,trim={6cm 0cm 6cm 0.8cm},clip]{figs/experimental_sys_uncs}
		\end{varblock}
		\begin{varblock}[5.9cm]{\centering Theoretical Uncertainties}
			\centering
			\includegraphics[scale=0.39,trim={4cm 0cm 4cm 0.8cm},clip]{figs/theoretical_sys_uncs}
		\end{varblock}
	\end{multicols}
	\begin{itemize}
		\justifying
		\item The systematic uncertainties on the VBF ANN discriminants are added into the statistical analysis via their nominal and shifted shapes;
	\end{itemize}
\end{frame}

\begin{frame}{ANN Discriminants Systematic Uncertainties}
	\begin{itemize}
		\justifying
		\item Systematic uncertainties on the ANN discriminants have been estimated from two sources: the systematic uncertainty on their inputs and the systematic uncertainty on the 3$^{rd}$ jet;
		\item The first case (which affect both jet-based categories) the systematic uncertainty on the ANN discriminants shape and yield has been derived by feeding the discriminants with the $\pm$1$\sigma$ shifted value of each input;
		\item The shifts are produced from one input variable at a time, such that, in the end there are $N_{(Inputs)} \times [1+2.N_{(InputsUncertainties)}]$ ANN distributions, including the nominal and shifted shapes;
		\item This procedure follows similar idea applied in previous CMS analysis:
		\begin{itemize}
		\justifying
			\item \url{cms.cern.ch/iCMS/jsp/openfile.jsp?tp=draft&files=AN2012_141_v9.pdf}
			\item \url{https://cds.cern.ch/record/2205282}
			\item \url{https://cds.cern.ch/record/2273847}
		\end{itemize}
	\end{itemize}
\end{frame}

\begin{frame}{ANN Discriminants Systematic Uncertainties}
	\begin{itemize}
		\justifying
		\item Here is an example of nominal and shifted distributions (superimposed) from one ANN for qqH and ggH (largest background) processes;
		\item The shifts look good and under control (in other words the discriminants are stable), mainly for the signal;
	\end{itemize}
	\center
	\vspace{-0.2cm}
	\begin{overpic}
		[scale=0.45,trim={7cm 0cm 0cm 0cm},clip]{/home/micah/cernbox/MonoHiggsHZZ4L/KerasV5/Uncertainty/datacard_root_example}
		\put(80,75){\color{red}\textbf{qqH}}
		\put(20,75){\color{blue}\textbf{ggH}}
	\end{overpic}
\end{frame}

\begin{frame}{ANN Discriminants Systematic Uncertainties}
	\begin{itemize}
		\justifying
		\item Since this analysis is proposing the usage of the 3$^{rd}$ jet and is not possible:
		\begin{itemize}
		\justifying
			\item to replace the current VBF MC sample (VBF-H2J) by its NLO VBF-H3J version\footnote{Process VBF\_HJJJ available at PowhegV2 (private generation following 2016 configurations).};
			\item or merge the two MC samples in suitable way;
		\end{itemize}
		\item A systematic uncertainty\footnote{It affects only Njets3 category.} because of using the 3$^{rd}$ jet from the current VBF MC sample has been estimated by computing the ratio between the ANN distribution using VBF-H2J and VBF-H3J, separately per four-lepton channel:
	\end{itemize}
	\begin{multicols}{3}
		\begin{varblock}[3.8cm]{\centering $4\mu$}
			\includegraphics[scale=0.19,trim={1.5cm 1cm 3cm 1cm},clip]{figs/k24nj3_3rdJetUncertainty_4mu}
		\end{varblock}
		\begin{varblock}[3.8cm]{\centering $4e$}
			\includegraphics[scale=0.19,trim={1.5cm 1cm 3cm 1cm},clip]{figs/k24nj3_3rdJetUncertainty_4e}
		\end{varblock}		
		\begin{varblock}[3.8cm]{\centering $2e2\mu$}
			\includegraphics[scale=0.19,trim={1.5cm 1cm 3cm 1cm},clip]{figs/k24nj3_3rdJetUncertainty_2e2mu}
		\end{varblock}				
	\end{multicols}	
\end{frame}
		
\subsection{Statistical Analysis and Results}
\begin{frame}{Statistical Analysis}
	\begin{itemize}
		\justifying
		\item The statistical analysis was done using the Higgs Combine tool:
		\begin{itemize}
			\item Higgs Combine tool is a package developed by LHC statistic community;
			\item It allows one to compute cross-sections, limits and significances taking into account the statistical and systematic uncertainties;
			\item It is used in several analysis in the LHC experiments nowadays;
		\end{itemize}
	\end{itemize}
	\begin{itemize}
		\item Here, a 1D shape analysis has been done:
		\begin{itemize}
			\item Inputs were ANN discriminants via 1D histograms (with proper signal and background normalizations) and the statistical and systematical uncertainties;
			\item Results achieved by combining the ANN discriminants from each jet-based category are highlighted in the next slides;			
		\end{itemize}
	\end{itemize}
\end{frame}

\begin{frame}{Statistical Analysis}
	\begin{itemize}
		\justifying
		\item The VBF signal strength is measured to be $\mu \equiv \sigma^{Obs}_{qqH} / \sigma^{SM}_{qqH} = 1.28_{-0.84}^{+1.24}$ by combining ANN discriminants of Njets2 and Njets3 categories;
	\end{itemize}
	\vspace{-0.5cm}
	\begin{figure}
		\centering
		\subfloat[$\mu_{qqH}$ likelihood scans.]{\includegraphics[scale=0.2]{figs/LikelihoodScans}}
		\subfloat[$\mu_{qqH}$ best fit on each channel.]{\includegraphics[scale=0.2]{figs/ChannelsCompatibility}}
	\end{figure}
%	\vspace{-0.5cm}
%\begin{table}
%	\caption{Expected and observed signal strength modifiers from each category and four-lepton channel for 35.9fb$^{-1}$ of observed data at $\sqrt{s}=$ 13TeV.}
%	\centering
%	\footnotesize
%	\begin{tabular}{c|c|c|c|c|c|c|c}
%		\hline
%		Signal strength & $\mu_{qqH}^{4\mu,2J}$ & $\mu_{qqH}^{4e,2J}$ & $\mu_{qqH}^{2e2\mu,2J}$ & $\mu_{qqH}^{4\mu,3J}$ & $\mu_{qqH}^{4e,3J}$ & $\mu_{qqH}^{2e2\mu,3J}$ & $\mu_{qqH}^{4l,2J+3J}$\\
%		\hline
%		Expected & 1.00$^{+2.13}_{-1.49}$ & 1.00$^{+3.11}_{-1.00}$ & 1.00$^{+1.83}_{-0.96}$ & 1.00$^{+5.79}_{-1.00}$ & 1.00$^{+8.24}_{-1.00}$ & 1.00$^{+4.67}_{-1.00}$ & 1.00$^{+1.08}_{-0.70}$ \\
%		\hline
%		Observed & 0.00$^{+1.36}_{-0.00}$ & 0.00$^{+1.36}_{-0.00}$ & 3.10$^{+2.69}_{-1.79}$ & 3.13$^{+6.47}_{-2.98}$ & 0.00$^{+3.78}_{-0.00}$ & 1.10$^{+4.77}_{-1.14}$ & 1.28$^{+1.24}_{-0.84}$ \\ 
%		\hline
%	\end{tabular}
%\end{table}	
\end{frame}

\begin{frame}{Statistical Analysis}
	\begin{itemize}
		\justifying
		\item Limits and significances have been computed via the HybridNew method:
		\begin{itemize}
		\justifying
			\item Limits show that hypothesis of VBF events in the present analysis can't be excluded, setting $\mu_{qqH}^{Obs} < 3.8$ and $\mu_{qqH}^{Exp} < 1.7$ at 95$\%$CL;
			\item Significances obtained are $\sigma_{qqH}^{Obs} = 1.9$ and $\sigma_{qqH}^{Exp} = 1.8$;
		\end{itemize}
	\end{itemize}
	\begin{figure}
		\centering
		\subfloat[Upper limits on $\mu_{qqH}$.]{\includegraphics[scale=0.2]{figs/Limits}}
		\subfloat[Significances via the present analysis.]{\includegraphics[scale=0.2]{figs/Pvalues}}
	\end{figure}
\end{frame}

\begin{frame}{Projections for Future Luminosities}
	\begin{multicols}{2}
	\begin{minipage}{6cm}
	\begin{itemize}
		\justifying
		\item The VBF signal strength measurement and the significance of the present analysis has been projected for future luminosities scenarios at the LHC;
		\item Systematic uncertainties have been accounted by scaling the present luminosity;
		\item The total uncertainty on the measurement of $\mu_{qqH}$ is expected to reduce $\sim$87$\%$ at $L = 150$fb$^{-1}$;
		\end{itemize}
	\end{minipage}
	\begin{overpic}
	   [scale=0.2]{figs/ChannelsCompatibility_39fb_vs_150fb}
	   \put(35,90){35.9fb$^{-1}$ vs. 150fb$^{-1}$}
	\end{overpic}
	\end{multicols}
	\vspace{-0.5cm}
	\begin{itemize}
		\justifying
		\item Significance of 5.1$\sigma$ is expected for a luminosity 10x larger than the present one. Below is a scan of the expected VBF significance between the present luminosity (35.9fb$^{-1}$) and future scenarios at the LHC.
	\end{itemize}
	\vspace{-0.4cm}
	\begin{table}
	\centering
	\begin{tabular}{c|c|c|c|c|c|c|c}
		\hline
		Luminosity (fb$^{-1}$)       & 35.9 & 150.0 & 300.0 & 359.0 & 1077.0 & 1795.0 & 3000.0\\
		\hline
		Factor (x $L^{35.9fb^{-1}}$) & 1.00 & 4.18  & 8.36  & 10.00 & 30.00  & 50.00  & 83.57\\
		\hline
		Expected significance        & 1.8  & 3.4   & 4.7   & 5.1   & 8.6    & 10.9   & 14.0\\
		\hline
		\end{tabular}
	\end{table}	
\end{frame}

\begin{frame}{Most $\&$ Least VBF-Like Event Display in each Category}
\centering
\vspace{0.6cm}
\begin{overpic}
	[scale=0.22]{figs/event_display_Run278167_Lumi1564_Event2852832207_Njets2_ANN0p93_DoubleEG_Run2016F_2}
	\put(30,63){ANN score: 0.93}
	\put(80,65){Njets2 Category}
\end{overpic}
\begin{overpic}
	[scale=0.22]{figs/event_display_Run275658_Lumi216_Event425010503_Njets2_ANN0p15_DoubleEG_Run2016C_2}
	\put(30,63){ANN score: 0.15}
\end{overpic}\\[0.6cm]
\begin{overpic}
	[scale=0.22]{figs/event_display_Run278018_Lumi453_Event844012107_Njets3_ANN0p77_DoubleMuon_Run2016F_2}
	\put(30,63){ANN score: 0.77}
	\put(80,65){Njets3 Category}	
\end{overpic}
\begin{overpic}
	[scale=0.22]{figs/event_display_Run278018_Lumi361_Event673988874_Njets3_ANN0p04_DoubleEG_Run2016F_2}
	\put(30,63){ANN score: 0.04}
\end{overpic}
\end{frame}

\section{CMS L1 Tracking Trigger AM+FPGA}
\begin{frame}
	\center
	\begin{minipage}{10cm}
		\begin{varblock}[8cm]{}
			\center
			\huge \color{red} CMS Level 1 Tracking Trigger\\ Associative Memory + FPGA
		\end{varblock}
	\end{minipage}
\end{frame}

\subsection{Introduction}
\begin{frame}{Introduction}
	\begin{itemize}
		\justifying
		\item During his PhD the  author was involved into the CMS Level 1 Tracking Trigger AM+FPGA project:
		\begin{itemize}
		\justifying
			\item one of the three CMS-L1TT projects (CMS upgrade phase II - HLLHC):
			\begin{itemize}
		\justifying
				\item Inclusion of inner-tracker data as part of L1 trigger;
				\item Original tracker designed for $L_{inst.} \sim 10^{34}$ cm$^{-2}$.s$^{-1}$ and PU$_{Ave.} \sim$ 20-30;
				\item Expected in phase II: $L_{inst.} \sim 10^{34}$ cm$^{-2}$.s$^{-1}$ and PU$_{Ave.} \sim$ 140-200;
				\item Required decision time: 5$\mu$s;
				\item 500-1k Tb/s of data to be processed;
			\end{itemize}
			\item leaded by Fermilab working group;
			\item the project aims for the usage of Associative Memories in combination with FPGAs;
			\item the author studied and implemented new components on the available software:
			\begin{itemize}
		\justifying
				\item synthetic match;
				\item duplicate removal; 
				\item stub bending;
				\item road and combination truncation
				\item track fitter $\chi^{2}$ adjustment;
			\end{itemize}
			\item results have been produced with different high-lumi scenarios (2-10k events):
			\begin{itemize}
		\justifying
				\item ($\mu$/$\pi$/$e$)+PU(140,200,300,400);
				\item $\nu$+PU(140,200,250)  (simulates pure PU, low $p_{T}$ particles);
				\item $t\bar{t}$+PU200;
				\item jets($p_{T}=250GeV$)+PU200;
			\end{itemize}
			\item hardware work: board inspections and tests;
		\end{itemize}
	\end{itemize}
\end{frame}

\begin{frame}{The CMS-L1TT AM+FPGA Approach}
\begin{flushright}
	\includegraphics[scale=0.45]{CMSL1TTfigs/cms_l1TT_strategy}
\end{flushright}
\vspace{-1.8cm}
\begin{itemize}
	\begin{minipage}{5cm}
		\item The three CMS-L1TT projects can be divided into three stages:
	\end{minipage}
	\begin{itemize}
		\vspace{0.2cm}
		\begin{minipage}{6cm}
			\item \textbf{Data Formatting}: fragmentation of the CMS detector in $\eta-\phi$ sectors (trigger towers);
		\end{minipage}
		\begin{minipage}{10cm}
			\item \textbf{Pattern Recognition}: selection of coarse hits patterns (potentially real tracks);
		\end{minipage}
		\begin{minipage}{11cm}
			\item \textbf{Track Fitting}: extraction of refined track info using all hits from selected patterns;
		\end{minipage}
	\end{itemize}
\end{itemize}
\end{frame}

\begin{frame}{The CMS-L1TT AM+FPGA Approach}
	\begin{itemize}
		\justifying
		\item Here's the main idea and definitions adopted in the CMS L1TT AM+FPGA approach:
		\begin{itemize}
		\justifying
			\item \textbf{Superstrip (SS)}: cluster of hits in the detector layers. They receive an ID based on their $z-\phi$ position;
			\item \textbf{Road}: pattern of built from SS's;
		\end{itemize}
	\end{itemize}
	\begin{multicols}{2}
	\includegraphics[scale=0.4,trim={0cm 0.1cm 0cm 0cm},clip]{CMSL1TTfigs/AMFPGA_scheme2}		
		\begin{itemize}
		\justifying
			\item An AM chip containing a big set of roads simulated via MC triggers the roads observed in real data. The hits from detector layers are processed in parallel;
			\item Once real roads are triggered, a set of possible hits combinations are built (possible tracks);
			\vspace{0.8cm}
			\item A fit select which combination is a real tracker;
		\end{itemize}
	\end{multicols}
\end{frame}

\subsection{Hardware Work}
\begin{frame}{The Hardware for the CMS-L1TT AM+FPGA}
\vspace{-2cm}
\begin{overpic}
	[scale=0.4]{CMSL1TTfigs/AMFPGA_scheme}
	\put(0,-30){\includegraphics[scale=0.06]{CMSL1TTfigs/20160413_152420}}
	\put(36,-30){\includegraphics[scale=0.035]{CMSL1TTfigs/Foto-24-05-16-19-51-58}}
	\put(80,-23){\includegraphics[scale=0.04]{CMSL1TTfigs/CMSL1TTAMFPGA_setup}}
\end{overpic}
\end{frame}

\subsection{Simulation Studies}
\begin{frame}{Simulation Studies: Stub Bending ($\Delta S$)}
	\begin{multicols}{2}
	\begin{itemize}
		\justifying
		\item The stub bending is the core idea behind the CMS L1TT project:
		\begin{itemize}
		\justifying
			\item It helps to mitigate PU (mainly low $p_{T}$ particles);
		\end{itemize}
		\item In the AM+FPGA approach the $\Delta S$ prevents random patterns to be fired:
		\begin{itemize}
		\justifying
			\item Without $\Delta S$ an AM pattern can be triggered by hits coming from different real tracks crossing the detector layers in different angles:
		\end{itemize}
		\item The $\Delta S$ was encoded in the AM framework via the SS ID's. The following formula defines the SS ID when the stub bending is required:\\
		\begin{equation}
			\nonumber
			ss = i_{\Delta S} * N_{\phi} + i_{\phi}
		\end{equation}
	\end{itemize}
	\centering
	\includegraphics[scale=0.4]{CMSL1TTfigs/stub_definition}\\[0.2cm]
	\includegraphics[scale=0.35,trim={20.5cm 2cm 0cm 2cm},clip]{CMSL1TTfigs/deltaS_approach}	
	\end{multicols}	
\end{frame}

\begin{frame}{Simulation Studies: Stub Bending ($\Delta S$)}
	\begin{itemize}
		\justifying
		\item The SS-$\Delta S$ formula:\\[-0.5cm]
	\begin{equation}
		\nonumber
		ss = i_{\Delta S} * N_{\phi} + i_{\phi}
	\end{equation}
	\begin{itemize}
		\justifying
		\item $i_{\Delta S}$: $\Delta S$ value of a given stub (max );
		\item $N_{\phi}$: number of trigger-tower segmentations in $\phi$;
		\item $i_{\phi}$: index of the $\phi$ segment which the stub belongs;
	\end{itemize}
	\item Two possibilities of building the SS ID's according to the $\Delta S$ values:
	\end{itemize}
	\vspace{0.2cm}
	\begin{multicols}{2}
		\centering
		\begin{overpic}
			[scale=0.3,trim={0cm 0cm 9cm 0cm},clip]{CMSL1TTfigs/ttbar_pu200_sym115577}
			\put(10,102){\textbf{Symmetric} (eg. SYM115577)}
		\end{overpic}
		\begin{overpic}
			[scale=0.3,trim={0cm 0cm 9cm 0cm},clip]{CMSL1TTfigs/ttbar_pu200_asym115577}
			\put(10,102){\textbf{Asymmetric}  (eg. ASYM115577)}
		\end{overpic}
	\end{multicols}	
\end{frame}

\begin{frame}{Simulation Studies: Stub Bending ($\Delta S$)}
	\begin{itemize}
		\justifying
		\item This $\Delta S$ approach allows the following schemes (negative ranges omitted):\\
	\begin{tabular}{c|c|l}
		\hline
		\hline
		$\#$ranges & range width & $\Delta S$ values ({\color{red}[~]} central ranges)\\
		\hline
		3 & 9 & {\color{red}[}-2.0, 2.0{\color{red}]}, [2.5, ...]\\
		5 & 7 & {\color{red}[}-1.5, 1.5{\color{red}]}, [2.0, 5.5], [6.0, ...]\\
		7 & 5 & {\color{red}[}-1.0, 1.0{\color{red}]}, [1.5, 3.5], [4.0, 6.0], [6.5, ...]\\
		9 & 3 & {\color{red}[}-0.5, 0.5{\color{red}]}, [1.0, 2.0], [2.5, 3.5], [4.0, 5.0], [5.5, ...]\\
		\hline 
	\end{tabular}
	\item Effect of $\Delta S$ on the number of roads and combinations:
	\begin{itemize}
		\justifying
		\item Reduction of up to $\sim$10x on roads and $\sim$50x on combinations;
		\item Symmetric method produces few more combs/roads than Asymmetric one;
	\end{itemize}
	\end{itemize}
	\centering
	\begin{overpic}
		[scale=0.3,trim={0cm 0cm 0cm 1cm},clip]{CMSL1TTfigs/ttbar_pu200_smu_roads_combs_deltaS}
		\put(15,6){$\mu @ t\bar{t}+PU200$}
	\end{overpic}
	\quad
	\begin{overpic}
		[scale=0.3,trim={0cm 0cm 0cm 1cm},clip]{CMSL1TTfigs/jet_Pt250_pu200_roads_combs_deltaS}
		\put(15,6){$jets(250GeV)+PU200$}
	\end{overpic}
\end{frame}

\begin{frame}{Simulation Studies: Stub Bending ($\Delta S$)}
	\begin{itemize}
		\justifying
		\item Effect of $\Delta S$ on the road efficiency:
		\begin{itemize}
		\justifying
			\item Up to 20$\%$ and 50$\%$ of efficiency can be recovered when truncation is applied;
		\end{itemize}
	\end{itemize}
	\centering
	\begin{overpic}
		[scale=0.4,trim={0cm 0cm 0cm 1cm},clip]{CMSL1TTfigs/ttbar_pu200_pt_split_graph}
		\put(15,-5){$\mu @ t\bar{t}+PU200$}
	\end{overpic}
	\quad
	\begin{overpic}
		[scale=0.4,trim={0cm 0cm 0cm 1cm},clip]{CMSL1TTfigs/jet250_pu200_pt_split_graph}
		\put(15,-5){$jets(250GeV)+PU200$}
	\end{overpic}
\end{frame}

\begin{frame}{Simulation Studies: Stub Bending ($\Delta S$) - The Edge of the Montain}
	\vspace{-0.2cm}
	\begin{multicols}{2}
	\begin{itemize}
		\justifying
		\item At the ending of author's iteration with the CMS L1TT AM+FPGA there was a worry about PU spikes (as it happened in LHC Run I);
		\item Studies with single-$\mu$+PU presented in the group showed large efficiency loss at very high PU:\\
		\centering
		\includegraphics[width=4cm,height=4cm]{CMSL1TTfigs/luciano_mupu_sf0p8_plot}\\
		\item The author decided to check that and apply the $\Delta S$ approach (not considered by them at that time):\\
		\centering
		\includegraphics[scale=0.34]{CMSL1TTfigs/muon_road_eff_vs_pile_up_sf0p6}\\
		\includegraphics[scale=0.32]{CMSL1TTfigs/muon_road_eff_vs_pile_up_deltaS}
	\end{itemize}
	\end{multicols}	
\end{frame}

\begin{frame}{Simulation Studies: Synthetic Efficiency}
	\begin{itemize}
		\justifying
		\item Synthetic efficiency is meant to check the efficiency based on the track parameters ($q/p_{T}$, $\phi_{0}$, $z_{0}$ and $cot~\theta$);
		\item Task: match MC and AM reco tracks using their parameters;
		\item For so, one defines a $\chi^{2}$-like function:
		\begin{equation}
			\nonumber
			\chi^{2}_{match} = \sum_{i=0}^{4} \frac{\delta^{2}p_{i}}{\Omega^{2}(q/p_{T})_{i}}, 
			\quad
			\delta p_{i} = (p_{i}^{MC}-p_{i}^{Reco})
		\end{equation}
		\item The $\Omega$ function normalizes the dependence between the resolution in each track parameter and $q/p_{T}$:
	\end{itemize}
	\centering
	\begin{overpic}
		[scale=0.16]{CMSL1TTfigs/r_qbpT_fit_single_pion_nopu}
	\end{overpic}
	\begin{overpic}
		[scale=0.16]{CMSL1TTfigs/r_phi0_fit_single_pion_nopu}
	\end{overpic}
	\begin{overpic}
		[scale=0.16]{CMSL1TTfigs/r_z0_fit_single_pion_nopu}
	\end{overpic}
	\begin{overpic}
		[scale=0.16]{CMSL1TTfigs/r_cotTheta_fit_single_pion_nopu}
	\end{overpic}			
\end{frame}

\begin{frame}{Simulation Studies: Synthetic Efficiency}
	\begin{itemize}
		\justifying
		\item The $\chi_{match}$ establishes three types of tracks when receives a threshold value $\bar{q}$:
		\begin{itemize}
		\justifying
			\item \textbf{Good}: first reco track with smallest $\chi_{match}$ < $\bar{q}$;
			\item \textbf{Duplicate}: other reco tracks with $\chi_{match}$ < $\bar{q}$;
			\item \textbf{Fake}: any reco track with $\chi_{match}$ $\geq$ $\bar{q}$;
		\end{itemize}
		\item In order to define the value $\bar{q}$ a tracking match based on stubs and the synthetic efficiency were simultaneously done:
		\begin{itemize}
		\justifying
			\item Scanning the cut on $\chi_{match}$ one checks (on a dedicated MC sample) the number of original stubs composing the reco track;
			\item Then, one checks which cut reduces the Fake rate and increases the Good rate as much as possible and, avoiding random stub combinations (GOOD <5S);
			\item It was decided to have {\color{blue}$\chi_{match} = 40$};
		\end{itemize}
	\end{itemize}
	\centering
	\includegraphics[scale=0.35]{CMSL1TTfigs/jet250pu200_synthetic_view}
	\quad
	\includegraphics[scale=0.35]{CMSL1TTfigs/jet250pu200_analytic_view}	
\end{frame}	
	
\begin{frame}{Simulation Studies: Duplicate Removal}
	\begin{itemize}
		\justifying
		\begin{multicols}{2}
		\item The pattern match based on SS clusters produces several duplicate tracks in the CMS L1TT AM+FPGA;
		\item For that reason, a procedure to remove such tracks has been developed: the duplicate removal;\\
		\includegraphics[scale=0.2]{CMSL1TTfigs/tracks_duplication}
		\end{multicols}		
		\item The duplicate removal is a stub-based mechanism with the following algorithm:
		\begin{itemize}
		\justifying
			\item[1] A reco track is taken from the reco tracks list (\textbf{A}) and inserted on a new tracks list (\textbf{A});
			\item[2] Then, a loop is done over the remaining tracks on list \textbf{A}:
			\begin{itemize}
		\justifying
				\item If a track is found to share a given number $\bar{n}$ of stubs with any track on the list \textbf{B}, it is removed from list \textbf{A};
				\item Otherwise, the track is stored into the list \textbf{B};
			\end{itemize}
			\item[3] The tracks remaining in the list \textbf{B} are the final tracks;
		\end{itemize}
		\item The DR mechanism was studied in order to tune the minimum number of stubs which allows massive remotion of duplicated tracks and high synthetic efficiency;
	\end{itemize}
\end{frame}	

\begin{frame}{Simulation Studies: Duplicate Removal}
	\begin{itemize}
		\justifying
		\item Here are some results obtained for the DR tunning\footnote{\tiny Notice the gap between the track and synthetic efficiencies: that comes from the extra stubs which builds up a good (stubs) combination for the original track};
		\item The final DR cut was chosen to be 0 (zero);
	\end{itemize}
	\begin{table}
	\small
	\centering
	$\mu+PU200$\\
	\begin{tabular}{c|c|c|c|c|c}
		\hline
		\hline
		DR option & Goods & Duplicates & Fakes & Track eff & Synthetic eff\\
		\hline
		None      & 1.976 &	25.785	   & 0.614 & 0.985	   & 0.989\\
		5         & 1.976 &	25.785	   & 0.614 & 0.985	   & 0.989\\
		4         & 1.976 &	8.898	   & 0.275 & 0.98	   & 0.989\\
		3         & 1.973 &	0.604	   & 0.095 & 0.964	   & 0.989\\
		2         & 1.969 &	0.065	   & 0.047 & 0.953	   & 0.989\\
		1         & 1.967 &	0.007	   & 0.039 & 0.951	   & 0.988\\
		\rowcolor{light_red}
		0         & 1.966 &	0.000	   & 0.038 & 0.951	   & 0.988\\
		\hline
	\end{tabular}
	\\[0.2cm]
	$jet(p_{T}=250GeV)+PU200$\\
	\begin{tabular}{c|c|c|c|c|c}
		\hline
		\hline
		DR option & Goods & Duplicates & Fakes & Track eff & Synthetic eff\\
		\hline
		None	  & 8.506 &	143.735	   & 8.924 & 0.89	   & 0.897\\
		5	      & 8.506 &	143.735	   & 8.924 & 0.89	   & 0.897\\
		4	      & 8.506 &	52.935	   & 4.109 & 0.883	   & 0.897\\
		3	      & 8.481 &	4.746	   & 1.167 & 0.823	   & 0.895\\
		2	      & 8.431 &	0.642	   & 0.597 & 0.754	   & 0.889\\
		1	      & 8.412 &	0.067	   & 0.506 & 0.738	   & 0.887\\
		\rowcolor{light_red}
		0	      & 8.406 &	0.003	   & 0.482 & 0.74	   & 0.886\\
		\hline
	\end{tabular}
	\end{table}	
\end{frame}

\begin{frame}{Simulation Studies: Final FOMs}
	\begin{itemize}
		\justifying
		\item The next slides summarizes the final results found via the simulation package adopting the implementations presented here;
		\item The FOMs (figures of merit) are the common graphs used within the CMS L1TT AM+FPGA approach in order to show the performance of simulation studies;
		\item The FOMs are the efficiency and track categorization rates versus ($p_{T}$, $\eta$, $\phi$);
	\end{itemize}
\end{frame}	

\begin{frame}{Simulation Studies: Final FOMs - $\mu+PU300$}
	Track reconstruction efficiency for $\mu+PU300$ sample. The pattern bank used had 64k patterns and truncation at 200 roads and 500 combinations has been applied. Duplication removal was applied by requiring DR=0.
	\begin{figure}
		\centering
		\begin{overpic}
			[scale=0.16,trim={1cm 0cm 1cm 1cm},clip]{CMSL1TTfigs/final_plots/muPU300/mu_pu300_track_eff_pt_trunc}
		\end{overpic}
		\begin{overpic}
			[scale=0.16,trim={1cm 0cm 1cm 1cm},clip]{CMSL1TTfigs/final_plots/muPU300/mu_pu300_track_eff_eta_trunc}
			\put(25,65){\color{blue}\tiny$\epsilon_{synthetic}(p_{T}>3GeV)$ = 0.36}
			\put(25,60){\color{red}\tiny$\epsilon_{synthetic}(p_{T}>3GeV)$ = 0.95}
			\put(25,55){\color{violet}\tiny$\epsilon_{synthetic}(p_{T}>3GeV)$ = 0.94}
		\end{overpic}
		\begin{overpic}
			[scale=0.16,trim={1cm 0cm 1cm 1cm},clip]{CMSL1TTfigs/final_plots/muPU300/mu_pu300_track_eff_phi_trunc}
		\end{overpic}\\	
		\begin{overpic}
			[scale=0.16,trim={1cm 0cm 1cm 1cm},clip]{CMSL1TTfigs/final_plots/muPU300/mu_pu300_track_ratio_pt_trunc}
		\end{overpic}
		\begin{overpic}
			[scale=0.16,trim={1cm 0cm 1cm 1cm},clip]{CMSL1TTfigs/final_plots/muPU300/mu_pu300_track_ratio_eta_trunc}
		\end{overpic}	
		\begin{overpic}
			[scale=0.16,trim={1cm 0cm 1cm 1cm},clip]{CMSL1TTfigs/final_plots/muPU300/mu_pu300_track_ratio_phi_trunc}
		\end{overpic}	
	\end{figure}	
\end{frame}

\begin{frame}{Simulation Studies: Final FOMs - $\mu+PU400$}
	Track reconstruction efficiency for $\mu+PU400$ sample. The pattern bank used had 64k patterns and truncation at 200 roads and 500 combinations has been applied. Duplication removal was applied by requiring DR=0.	
	\begin{figure}
		\centering
		\begin{overpic}
			[scale=0.16,trim={1cm 0cm 1cm 1cm},clip]{CMSL1TTfigs/final_plots/muPU400/mu_pu400_track_eff_pt_trunc}
		\end{overpic}
		\begin{overpic}
			[scale=0.16,trim={1cm 0cm 1cm 1cm},clip]{CMSL1TTfigs/final_plots/muPU400/mu_pu400_track_eff_eta_trunc}
			\put(25,45){\color{blue}\tiny$\epsilon_{synthetic}(p_{T}>3GeV)$ = 0.36}
			\put(25,40){\color{red}\tiny$\epsilon_{synthetic}(p_{T}>3GeV)$ = 0.95}
			\put(25,35){\color{violet}\tiny$\epsilon_{synthetic}(p_{T}>3GeV)$ = 0.94}
		\end{overpic}
		\begin{overpic}
			[scale=0.16,trim={1cm 0cm 1cm 1cm},clip]{CMSL1TTfigs/final_plots/muPU400/mu_pu400_track_eff_phi_trunc}
		\end{overpic}\\	
		\begin{overpic}
			[scale=0.16,trim={1cm 0cm 1cm 1cm},clip]{CMSL1TTfigs/final_plots/muPU400/mu_pu400_track_ratio_pt_trunc}
		\end{overpic}
		\begin{overpic}
			[scale=0.16,trim={1cm 0cm 1cm 1cm},clip]{CMSL1TTfigs/final_plots/muPU400/mu_pu400_track_ratio_eta_trunc}
		\end{overpic}	
		\begin{overpic}
			[scale=0.16,trim={1cm 0cm 1cm 1cm},clip]{CMSL1TTfigs/final_plots/muPU400/mu_pu400_track_ratio_phi_trunc}
		\end{overpic}	
	\end{figure}	
\end{frame}


\section{The Fast Matrix Element (FastME)}
\begin{frame}
	\center
	\begin{minipage}{10cm}
		\begin{varblock}[6.5cm]{}
			\center
			\huge \color{red} A Fast Matrix Element
		\end{varblock}
	\end{minipage}
\end{frame}

\subsection{Introduction}
\begin{frame}{Theoretical Foundation}
	\begin{itemize}
		\justifying
		\item Here is presented a procedure called \textit{Fast Matrix Element} (\textit{Fast}ME);
		\item It was studied in the very beginning of the author's PhD;
		\item The project was first idealized by prof(s). Andre Sznajder (DFNAE-UERJ) and Stephen Mrenna (CSD - FNAL):
		\begin{itemize}
		\justifying
			\item A method capable of deriving event weight from MC sampling into a given phase space;
			\item It should allow one to get proper normalization of random events, for instance;
		\end{itemize}
	\end{itemize}
	\begin{multicols}{2}
	\begin{varblock}[5.5cm]{\centering Why {\Huge Fast}?}
		\begin{table}
			Time to compute the weight per event via \textit{MadWeight}5. For an usual analysis these numbers multiply by thousand.\\
			\begin{tabular}{c|c}
				\hline
				Process                    & Time/Event (s)\\
				\hline
				ZH                         & $<$5\\
				$t\bar{t}$ fully-leptonic  &   10\\
				Zbb                        &   18\\
				$t\bar{t}$ semi-leptonic   &   41\\
				$t\bar{t}$H fully-leptonic &   60\\
				\hline
			\end{tabular}
		\end{table}		
	\end{varblock}
	\begin{varblock}[5.8cm]{\centering {\Huge ME} Methods}
	\begin{eqnarray}
		\nonumber
		\mathcal{P}(x|\alpha) &=& \frac{1}{\sigma_{\alpha}} \int d\omega_{1} d\omega_{2} f(\omega_{1}) f(\omega_{2})\\
		\nonumber
		&& \int d\Phi(y)|\mathcal{M}_{\alpha}(y)|^{2} W(x,y)
	\end{eqnarray}
	\vspace{-0.6cm}
	\begin{itemize}
		\justifying
		\item $\mathcal{P}(x|\alpha)$ is an event probability;
		\item $W(x,y)$ handled as approximation;
		\item $\mathcal{M}_{\alpha}(y)$ not possible for all physics;
		\item $\mathcal{M}_{\alpha}(y)$ in NLO or so on?
	\end{itemize}
	\end{varblock}
	\end{multicols}
\end{frame}

\begin{frame}{Theoretical Foundation}
	\begin{itemize}
		\justifying
		\item The original idea of finding event weights didn't lead to promising results: assigned weights didn't model properly the events;
		\item A new idea appeared, then: 
		\begin{itemize}
		\justifying
			\item Would it be possible to discriminate events based on a match between the particles from a probe event and a MC one?
		\end{itemize}
		\item \textit{Fast}ME algorithm:
		\begin{itemize}
		\justifying
			\item[1] Loop over the particles (i) from a MC event and match them to the particles (j) from a probe event according to
			\begin{equation}
				\nonumber
				R^{2}_{(i,j)} = \sum_{k=1}^{n}~ \left( \dfrac{v_{k}^{(i,MC)}-v_{k}^{(j,Data)}}{\sigma_{v_{k}}} \right)^2
			\end{equation}
			where, k stands for the kinematic variables ($p_{T}$, $\eta$, $\phi$) and the particles pairs $(i,j)$ are chosen to minimize $R_{i,j}$;
			\item[2] A distance between the probe event and the MC event is computed by summing in quadrature the minimum distances ($R_{i,j}$) between their particles:
			\begin{equation}
				\nonumber
				D^{2} = \sum_{i=1}^{m}~[R^{2}_{(i,j)}]_{min},~~\mathrm{with}~~j(i+1) ~!=~j(i)
			\end{equation}
			\item[3] Finally, a discriminant for the probe event is computed using the closest MC events (from each class) via the formulas
			\vspace{-0.5cm}
			\begin{multicols}{2}
				\begin{equation}
					\nonumber
					P_{SB}^{D} = \dfrac{D^{Bkg}_{Min}}{D^{Bkg}_{Min}+D^{Sig}_{Min}}
				\end{equation}\\
				\begin{equation}
					\nonumber
					P_{SB}^{W} = \dfrac{W^{Bkg}_{D_{min}}}{W^{Bkg}_{D_{min}}+W^{Sig}_{D_{min}}}
				\end{equation}
			\end{multicols}
		\end{itemize}
	\end{itemize}
\end{frame}

\begin{frame}{Theoretical Foundation}
	\begin{itemize}
		\justifying
		\item Here's an illustration of the method to clarify the algorithm. The MC events present a topology associated to the EM of a given physical process, such as, each particle has a correlation with the others particles in the event. An data event (black points) receives a probability of being from a kind or other (blue and red point) via the correlation of the distances (represented by the blue and red circles) between it and the MC events.
	\end{itemize}		
	\begin{figure}
	\flushleft
	\begin{overpic}
		[scale=0.4,trim={0cm 0cm 0cm 0cm},clip]{FastMEfigs/fastme_work_way}
		\put(60,77){\large $\phi$}
		\put(100,35){\large $\eta$}
		\put(66,29){{\color{red}$\bullet$}\hspace{0.07cm} class 1}
		\put(66,23){{\color{blue}$\bullet$}\hspace{0.07cm} class 2}
		\put(66,17){{\color{black}$\bullet$}\hspace{0.07cm} probe event}
		\put(65,10){{\color{red}$\bigcirc$} class 1 matches $[R^{2}_{(i,j)}]_{min}$}
		\put(65,2){{\color{blue}$\bigcirc$} class 2 matches $[R^{2}_{(i,j)}]_{min}$}
		\put(110,15){
			\begin{overpic}[scale=0.16]{FastMEfigs/psbD_vs_psbW}
				\put(5,71){\footnotesize $P_{SB}^{D} = \dfrac{D^{Bkg}_{Min}}{D^{Bkg}_{Min}+D^{Sig}_{Min}}$}
				\put(50,71){\footnotesize $P_{SB}^{W} = \dfrac{W^{Bkg}_{D_{min}}}{W^{Bkg}_{D_{min}}+W^{Sig}_{D_{min}}}$}
			\end{overpic}
		}
	\end{overpic}
	\end{figure}
\end{frame}

\subsection{Simulation Studies}
\begin{frame}{Simulation Studies: \textit{Fast}ME Applied to HZZ4L CMS Data}
	\begin{itemize}
		\justifying
		\item After interesting results with MadGraph/Powheg/Sherpa samples of ggH and qqZZ, it was natural an interest of applying \textit{Fast}ME to the HZZ4L data collected by CMS on 2015 during the LHC RunI;
		\item The results have been compared to the formal CMS discriminant, the so called MELA for discriminating SM Higgs against $qqZZ$ background;
		\item Below: (a) observed events classified as signal and background by the \textit{Fast}ME, (b) $P_{SB}^{D}$ and (c) MELA discriminants distribution versus the $m_{4l}$.
	\end{itemize}
	\begin{figure}
	\centering
	\begin{overpic}
		[width=3.8cm,height=4cm,trim={0cm 0cm 13cm 0cm},clip]{FastMEfigs/fastme_discriminant_and_m4l}
		\put(45,-8){(a)}
	\end{overpic}
	\begin{overpic}
		[width=3.8cm,height=3.8cm,trim={0cm 0cm 11.4cm 0cm},clip]{FastMEfigs/fastme_2D_m4l_comparison_to_mela}
		\put(50,-8){(b)}
	\end{overpic}
	\begin{overpic}
		[width=3.8cm,height=3.8cm,trim={11.6cm 0cm 0cm 0cm},clip]{FastMEfigs/fastme_2D_m4l_comparison_to_mela}
		\put(50,-8){(c)}
	\end{overpic}	
	\end{figure}	
\end{frame}

\subsection{The FastME Package}
\begin{frame}{The \textit{Fast}ME Package}
	\begin{itemize}
		\justifying
		\item The success of \textit{Fast}ME idea on real CMS data encouraged us to move the standalone codes created until that moment into a organized package;
		\item During the author's first travel to Fermilab, this package started to be maintained on GitHub:
	\end{itemize}
	\centering
	\includegraphics[scale=0.34]{FastMEfigs/fastme_github_page}	
\end{frame}

\begin{frame}{The End of \textit{Fast}ME Project}
	\begin{itemize}
		\item Although the modified original idea of \textit{Fast}ME has showed some nice results we faced two issues:
		\begin{itemize}
			\item It doesn't have a good performance for discriminating VBF against ggH;
			\item TMVA has a similar implementation and is more flexible;
		\end{itemize}		
		\includegraphics[width=5cm,height=5cm,trim={0cm 0cm 18cm 0cm},clip]{FastMEfigs/ggH_vs_VBF_test}
		\quad
		\includegraphics[width=6cm,height=4.5cm,trim={0cm 0cm 0cm 0cm},clip]{FastMEfigs/knn_tmva}
		\item Such issues lead to the end of the project:
		\begin{itemize}
			\item No power enough for the presented physics analysis;
			\item Not a new method (we re-invented the wheel);
		\end{itemize}
	\end{itemize}
\end{frame}

\section{Conclusions}
\begin{frame}
	\center
	\begin{minipage}{10cm}
		\begin{varblock}[5cm]{}
			\center
			\huge \color{red} CONCLUSIONS %\\ AND\\ FUTURE PLANS
		\end{varblock}
	\end{minipage}
\end{frame}


\begin{frame}{Conclusions: CMS-AN-18-120}
	\begin{itemize}
		\justifying
		\item {\color{blue}ANN approach successfully implemented for an isolated VBF $H \rightarrow ZZ \rightarrow 4l$ XS measurement};
		\item {\color{blue}Reliable} procedure implemented for {\color{blue}systematic uncertainties};
		\item {\color{blue}Results} provided from the combination of our {\color{blue}best ANN configurations}:
		\begin{itemize}
		\justifying
			\item best fit for signal strength: {\color{blue}$\mu_{qqH}^{Exp} = 1.00_{-0.70}^{+1.08}$} and {\color{blue}$\mu_{qqH}^{Obs} = 1.28_{-0.84}^{+1.24}$};
			\item 95$\%$CL upper limits on $\mu_{qqH}$: {\color{blue}$\mu_{qqH}^{Exp} < 1.66$} and {\color{blue}$\mu_{qqH}^{Obs} < 3.79$};
			\item significances: {\color{blue}$\sigma_{qqH}^{Exp} = 1.8$} and {\color{blue}$\sigma_{qqH}^{Obs} = 1.9$};
		\end{itemize}
		\item {\color{blue}Projections} provided for future luminosity scenarios at the LHC:
		\begin{itemize}
		\justifying
			\item Expected to improve signal strength precision up to {\color{blue}$\sim$87$\%$} at the end of RunII;
			\item Significance evolution:\\
			\begin{tabular}{c|c|c|c|c|c|c|c}
				\hline
				Luminosity (fb$^{-1}$) & 35.9 & 150.0 & 300.0 & 359.0 & 1077.0 & 1795.0 & 3000.0\\
				\hline
				Factor                 & 1.00 & 4.18  & 8.36  & 10.00 & 30.00  & 50.00  & 83.57\\
				\hline
				Expected significance  & 1.8  & 3.4   & 4.7   & 5.1   & 8.6    & 10.9   & 14.0\\
				\hline
			\end{tabular}
		\end{itemize}
		\item Analysis documentation is ready and released: {\color{blue}AN-18-120};
		\item {\color{blue}No issues} raised last meeting with HZZ subgroup (December 7$^{th}$, 2018) at CERN (green light for a thesis endorsement);
	\end{itemize}
\end{frame}

\begin{frame}{Conclusions: CMS-AN-18-120}
	\begin{itemize}
		\justifying
		\item {\color{red}Ongoing}: analysis of full {\color{blue}2017 MC $\&$ Data}:
		\begin{itemize}
		\justifying
			\item package of macros for future studies (parallel ANN training, Z+X derivation and statistical analysis);
			\item already in use by a colleague in Bari (Nicola's student).
		\end{itemize}
	\end{itemize}
	\centering
	\includegraphics[scale=0.45]{figs/github_AN_18_120}
\end{frame}

\begin{frame}{Conclusions: CMS L1TT AM+FPGA}
	\begin{multicols}{2}
	\begin{itemize}
		\item Significant contribution has been given to the CMS L1TT AM+FPGA:
		\begin{itemize}
			\item Several new MC samples generated and made available for the group;
			\item Developments, studies and implementations:
			\begin{itemize}
				\item Synthetic matching and efficiency;
				\item Duplicate removal;
				\item Stub bending ($\Delta S$);
				\item Effects of truncation on roads and/or combinations;
				\item Tracking fitter $\chi^{2}$ cut revision;
			\end{itemize}
			\item Support during the electronic inspections:
			\begin{itemize}
				\item Check up of Pulsar BII boards and the PRMs;
				\item Check up of optical cables connecting Pulsar boards;
				\item Check up of boards connected in the crates;
			\end{itemize}
		\end{itemize}
	\end{itemize}
	\begin{itemize}
		\justifying
		\item Creation of dedicated documentation about nomenclatures, workflow and implementations developed by the author within the package;
	\end{itemize}
	\centering
	\includegraphics[scale=0.5]{CMSL1TTfigs/notes_cover}
	\end{multicols}	
\end{frame}

\begin{frame}{Conclusions: Fast Matrix Element}
	\begin{itemize}
		\item Although the \textit{Fast}ME idea didn't get finished, some conclusions can be drawn from it:
		\begin{itemize}
			\item The results are a re-statement of the KNN method developed by the TMVA team and suggest that it can even be simplified (no need of a volume in the chosen phase space);
			\item This \textit{Fast}ME idea is quite sensitive to the physical process in analysis and can even become useless;
			\item The method is also sensitive to the size of the pattern banks used in the analysis and study of bias correction could be need;
			\item The method is easily applicable in several process, some of which the MEM can not or don't have higher order corrections included;
		\end{itemize}
	\end{itemize}
\end{frame}

\begin{frame}{}
	\centering
	\begin{minipage}{10cm}
		\begin{varblock}[10cm]{}
			\centering
			\begin{minipage}{9cm}
				\begin{varblock}[9cm]{}
					\centering
					\color{red}
					\Huge Thank you for the attention!
				\end{varblock}
			\end{minipage}
			\vspace{0.3cm}
		\end{varblock}
	\end{minipage}
\end{frame}

\begin{frame}
	\begin{center}
		\includegraphics[scale=0.5]{figs/backup}
	\end{center}
\end{frame}

\begin{frame}{Datasets used in the Analysis}
	\centering
	\includegraphics[scale=0.5,trim={0cm 0.5cm 0cm 0cm},clip]{figs/datasets}
\end{frame}

\begin{frame}{JETMET POG Filters}
	\centering
	\includegraphics[scale=0.6,trim={0cm 0cm 0cm 0.5cm},clip]{figs/met_filters}
\end{frame}

\begin{frame}{Post-Fit Yields and Distributions}
	\centering
	\includegraphics[scale=0.45]{figs/postfit_yields}\\[0.2cm]
	\includegraphics[scale=0.5]{figs/postfit_shapes}
\end{frame}

\begin{frame}{Comparison Between Best Fits of Njets2 and Njets3 Categories}
	\begin{itemize}
		\justifying
		\item The best fits from each jet-based category. As it is shown, there's no advantage in using them alone instead of combining as it was done in the analysis;
	\end{itemize}
	\centering
	\includegraphics[scale=0.25]{figs/ChannelCompatibility_ComparisonBetweenNjets2Njets3}
\end{frame}

\begin{frame}{Sensitivity of Combined ANNs}
	\begin{itemize}
		\justifying
		\item Note that, no systematic uncertainty has been accounted here;
	\end{itemize}
	\centering
	\includegraphics[scale=0.7,trim={2cm 9.5cm 2cm 0cm},clip]{figs/ann_comb_sensitivity}
	\includegraphics[scale=0.7,trim={2cm 1cm 2cm 8.6cm},clip]{figs/ann_comb_sensitivity}
\end{frame}

%\begin{frame}{ANNs Validation Plots}
%	\centering
%	\includegraphics[scale=0.7,trim={0cm 11.7cm 0cm 0cm},clip]{figs/ann_validation_plots}
%	\includegraphics[scale=0.7,trim={0cm 6cm 5.4cm 5.8cm},clip]{figs/ann_validation_plots}\\[0.5cm]
%	\includegraphics[scale=0.7,trim={5.4cm 6cm 0cm 5.8cm},clip]{figs/ann_validation_plots}
%	\includegraphics[scale=0.75,trim={0cm 0cm 0cm 11.5cm},clip]{figs/ann_validation_plots}
%\end{frame}

\begin{frame}{Events after SM Higgs Selections in each 4l Channel}
	\centering
	\includegraphics[scale=0.5]{figs/smhiggs_m4l_channels}
\end{frame}

\begin{frame}{Z+X Systematic Uncertainty and Final Yields}
	\begin{itemize}
		\justifying
		\item Z+X systematic uncertainty from FR: compute its variation by averaging MC ($D\varUpsilon$, $ZZ/Z\gamma$, $WZ$, $t\bar{t}$) FR and reweighing with 2P2F yields. Then propagate to Z+X estimation;
	\end{itemize}
	\centering
	\includegraphics[scale=0.7,trim={0cm 1.5cm 0cm 0cm},clip]{figs/FR_average_and_reweight_1D}
\end{frame}

\begin{frame}{Z+X Systematic Uncertainty and Final Yields}
	\begin{itemize}
		\justifying
		\item The behavior in 2D:
	\end{itemize}
	\centering
	\includegraphics[scale=0.6]{figs/FR_average_and_reweight}
\end{frame}

\begin{frame}{Z+X Systematic Uncertainty and Final Yields}
	\centering
	\includegraphics[scale=0.7]{figs/zx_final_yields}
\end{frame}

\begin{frame}{Systematic Uncertainties Impact on $\mu_{qqH}$ Fit}
	\centering
	\includegraphics[scale=0.5]{figs/impacts_obs_4l_combination}
\end{frame}

\begin{frame}{Architecture of Chosen ANNs}
\begin{figure}
	\caption{ANNs architecture created in this analysis. Black dots stand for inputs/neurons, while lines stand for the size of the ANN parameters chosen after training them. Wider and brighter lines means the parameter (weight or bias) associated to a given input for a neuron is larger (in other words its contribution is more relevant). ANN for Njets2 has 21:13:8 hidden neurons, while for Njets3 it has 11:9:7 (all neurons of SeLU type).}
	\subfloat[Njets2 - 21:13:8]{\includegraphics[width=6cm,height=5cm]{figs/k57nj2_architecture_horizontal}}
	\subfloat[Njets3 - 11:9:7]{\includegraphics[width=6cm,height=5cm]{figs/k24nj3_architecture_horizontal}}
\end{figure}
\end{frame}

\begin{frame}{Keras ModelCheckpoint Option Against Overfitting}
	\footnotesize
	\begin{itemize}
		\item Here is an example of overfitting in my case;
		\item Good solution found: \textbf{ModelCheckPoint}, it saves the model every time the validation loss decreases;
		\item Although the learning curves show overfitting the final ANN shapes from training and testing sub-datasets matches nicely;
	\end{itemize}
	\centering
	\includegraphics[width=6cm,height=6cm]{figs/TrainingHistoryLossOverfitting}
	\includegraphics[width=6cm,height=6cm]{figs/NNOvertrainingCheckOverfitting}
\end{frame}

\begin{frame}{Simulation Studies: Roads and Combinations Truncation}
	\begin{multicols}{2}
		\begin{itemize}
			\justifying
			\item It was needed to check the approach under truncation of roads and combinations (for reducing latency, for instace);
			\item Implemented two new flags into the software for controlling the number of roads and combinations to be accepted for further processing;
			\item A study done by the author showed the smallest impact on the efficiency due to truncation happens when roads are sorted by frequency:
		\end{itemize}
		\includegraphics[scale=0.4]{CMSL1TTfigs/ttbar_pu200_mu_road_sorting_comparison}
	\end{multicols}
\end{frame}

\begin{frame}{Simulation Studies: $\chi^{2}$ Revision}
	\begin{itemize}
		\justifying
		\begin{multicols}{2}		
			\item In order to optimize the final results, it was decided to re-study the tracking fitter $\chi^{2}$ value;
			\item It decides weather a combination of stubs is fine or not;
			\item Points to address:
			\begin{itemize}
				\justifying
				\item Just an unique cut applied;
				\item Dependence on the tracks $p_{T}$;
			\end{itemize}
			\flushright
			\begin{overpic}
				[scale=0.25]{CMSL1TTfigs/tfChi2_Pt_all}
			\end{overpic}
		\end{multicols}
		\item A new set of cuts have been adopted:
		\begin{itemize}
			\justifying
			\item 6/6 combinations have a tight and unique threshold given by the 99$\%$ percentile of the theoretical $\chi^{2}_{ndof=8}$ curve (equals to 20.2);
			\item 5/6 combinations have a $p_{T}$-based cut according to 8 ranges, which were defined to guarantee $\epsilon^{6/6+5/6}_{tracks} = 0.99 * \epsilon^{6/6+5/6}_{roads}$
		\end{itemize}
	\end{itemize}
\end{frame}

\begin{frame}{Simulation Studies: Bias from Pattern Bank Size}
	\begin{itemize}
		\justifying
		\item The first point addressed during the development of the project was the influence of the pattern banks size. On graphs (a) and (b) the background pattern bank has a fixed size while the signal one is varied. On graphs (c) and (d) the opposite case is shown. Note, here "purity" is computed using the absolute number of events (without normalization).
	\end{itemize}
	\begin{figure}
		\begin{overpic}
			[width=8cm,height=3cm,trim={0cm 0cm 0cm 0cm},clip]{FastMEfigs/purity_keeping_bkg_varying_sig}
			\put(25,20){(a)}
			\put(78,20){(b)}
		\end{overpic}\\[0.2cm]
		\begin{overpic}
			[width=8cm,height=3cm,trim={0cm 0cm 0cm 0cm},clip]{FastMEfigs/purity_keeping_sig_varying_bkg}
			\put(25,20){(c)}
			\put(78,20){(d)}		
		\end{overpic}
	\end{figure}	
\end{frame}

\begin{frame}{Simulation Studies: Impact of $\phi$ Variable}
	\begin{itemize}
		\justifying
		\item Some studies have been done in order to optimize the performance of the discriminants;
		\item In the beginning of the project results showed that $\phi$ and $E$ (energy) doesn't contribute and can actually worse the discriminant performance;
		\item Below is a comparison between the $P_{SB}^{D}$ distribution using $\phi$ and without it:
		\begin{figure}
			\begin{overpic}
				[scale=0.2,trim={0cm 0cm 0cm 2.3cm},clip]{FastMEfigs/comparacao_psbD_using_nousing_dPhi_scaledPt70}
				\put(10,47){\color{red}Signal}
				\put(10,44){\color{blue}Background}		
			\end{overpic}
		\end{figure}			
		\item Based on such plot $\phi$ was removed from the default algorithm within $\textit{Fast}$ME and left as an option to the user;
	\end{itemize}
\end{frame}

\begin{frame}{Simulation Studies: Scaling $v_{k}$'s Contribution}
	\begin{itemize}
		\justifying
		\item Another point of optimization was the scaling of the variables used to compute the $R_{(i,j)}$;
		\item Since the variables $v_{k}$ present quite different ranges of variation it's important do level them;
		\item Here's a scan showing the variation of the fraction of signal events being mis-classified as background in function of the $\delta p_{T}$. A fixed value of $\delta \eta = 5.0$ was used;
		\begin{figure}
			\begin{overpic}
				[scale=0.2,trim={0cm 0cm 0cm 0cm},clip]{FastMEfigs/fastme_sensibility_vs_dpTscaleFactor}
			\end{overpic}
		\end{figure}
		\item The latest version of the project has an automated method which assigns the cumulative mean of MC events as the scaling factors;
	\end{itemize}
\end{frame}

\begin{frame}{The \textit{Fast}ME Package}
	\begin{multicols}{2}
		\begin{itemize}
			\justifying
			\item Features:
			\begin{itemize}
				\begin{minipage}{5cm}
					\item User gives to the program a configuration file;
					\item Original samples are replicated;
					\item $R_{(j,i)}$ can be simultaneously computed for each class;
					\item Creates a ROOT file containing $D_{min}^{class}$ and $P_{SB}^{D/W}$;
				\end{minipage}			
			\end{itemize}
		\end{itemize}
		\centering
		\begin{overpic}
			[scale=0.4,trim={0cm 4.1cm 0cm 7cm},clip]{FastMEfigs/fme_config_card}
			\put(0,43){\scriptsize \textit{Fast}ME configuration file example}
		\end{overpic}
		\begin{overpic}
			[scale=0.2,trim={0cm 25cm 0cm 0cm},clip]{FastMEfigs/fme_running}
			\put(0,101){\scriptsize \textit{Fast}ME workflow view}
		\end{overpic}\\:\\
		\begin{overpic}
			[scale=0.2,trim={0cm 0cm 0cm 40cm},clip]{FastMEfigs/fme_running}
			\put(76,60){\includegraphics[scale=0.11]{FastMEfigs/Discriminant_Signal_vs_Background}}
			\put(76,0){\includegraphics[scale=0.11]{FastMEfigs/FastMatrixElement_ROC_Curve}}
		\end{overpic}
	\end{multicols}	
\end{frame}

%-------------------------------------------------------------
\end{document}
